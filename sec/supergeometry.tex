
\section{Supergeometric sketches}

A \( (\Z/2\Z) \)-graded vector space \( V = V_0 \oplus V_1 \) is referred to as a \defn{superspace} and the involution
\[
(\Pi V)_0 = V_1
\qquad
(\Pi V)_1 = V_0
\]
as \defn{parity change}.

A \defn{supercommutative algebra} is a superspace together with an associative product such that for all homogeneous elements \( a, b \) one has
\[
a b = (-1)^{|a||b|} b a .
\]
For example, \( C^\infty_{\R^m} \ot \bigwedge \R^n \) is a supercommutative algebra for any pair of non-negative integers \( m | n \).

%Similarly, for any superspace \( V \) its \defn{$n^\th$~symmetric power} is
%\begin{equation}\label{eq:SymPower}
%	\mathrm{Sym}^n V \defeq (V^{\otimes n})_{\sym_n} \cong
%	\quad \mathclap{\bigoplus_{0 \leq k \leq n} } \quad \big( \mathrm{Sym}^k V_{\bar{0}} \otimes {\textstyle \bigwedge^{n-k}} V_{\bar{1}} \big)
%\end{equation}
%and the canonical product in \( \mathrm{Sym} V = \bigoplus_n  \mathrm{Sym}^n V\) defines a supercommutative algebra.

\medskip
A \defn{ringed space} \( (X, \cO_X) \) is a topological space \( X \) together with a sheaf of rings \( \cO_X \) on \( X \).
It is a \defn{locally ringed space} if all stalks are local rings (that is, rings with a unique maximal ideal).

A smooth \( m \)-manifold can be defined as a Hausdorff, second-countable locally ringed space \( (M, \cO) \) such that
for every \( x \in M \) there exists an open neighborhood \( U \subset M \) and an open set \( U_0 \subset \R^m \)
with an isomorphism of locally ringed spaces
\[
(U, \cO|_U) \cong (U_0, C^\infty_{U_0}),
\]
where \( C^\infty_V \) is the sheaf of smooth \( \R \)-valued functions on \( V \).

\medskip
An \defn{\( (m | n) \)-supermanifold} is a locally ringed space \( (M, \cO) \)
where \( \cO \) is a sheaf of supercommutative \( \R \)-algebras such that every point \( x \in M \) has an open neighborhood \( U \subset X \) with an isomorphism of locally ringed spaces
\[
(U, \cO|_U ) \; \cong \; ( U_0, \; C^\infty_{U_0} \otimes_\R \textstyle\bigwedge \R\set{\theta^1,\dots,\theta^n} )
\]
where \( U_0 \subset \R^m \) is open and \( \bigwedge \R\set{\theta^1,\dots,\theta^n} \) is the Grassmann algebra on \( n \) generators, the \defn{odd coordinates}.
The space \( M \) is a smooth manifold, referred to as the \defn{body} of the supermanifold \( \cM \).
The sheaf \( \cO \) is called the \defn{sheaf of functions} of \( \cM \) and global sections are denoted \( C^\infty(\cM) \).

%The supermanifold denoted by \( \R^{m|n} \) is the supermanifold with body \( \R^m \) and sheaf of functions \( C^\infty_{\R^m} \ot \bigwedge \R^n \).

%The complex of differential forms on a smooth manifold \( M \) gives rise to a supermanifold with body \( M \) and sheaf of functions given on a small enough neighborhood \( U \) of \( x \) by \( C^\infty_U \ot \textstyle\bigwedge T_x^\ast M \).

\medskip
Given any vector bundle \( E \to M \), we can naturally construct a supermanifold \( \Pi E \) with functions \( C^\infty(\Pi E) \) isomorphic to \( \Gamma(\bigwedge E^\ast) \), the global sections of the Grassmann bundle of its linear dual.

From this perspective, differential forms and polyvector fields on a smooth manifold \( M \) can be written respectively as:
\[
\Omega(M) \cong C^\infty(\Pi\rT M)
\quad\text{and}\quad
\fX(M) \cong C^\infty(\Pi\rT^*\!M).
\]

\noindent\hrule

Under the first identification, the de Rham differential acts as a derivation of the ring of functions of \( \Pi\rT M \).
That is to say, it corresponds to a \defn{supervector field} \( Q \).
We can express the nipotency of the differential using the Lie bracket of supervector fields: \( [Q, Q] = 0 \).

\subsection{Derived brackets}

\begin{itemize}
	\item We have the graded manifold \(T^{*}[1]M\).
	\item Functions on it: \(C^{\infty}\!\big(T^{*}[1]M\big) \cong \Gamma\!\big(\Lambda^{\bullet} TM\big)\).
	\item Canonical bracket: \(\{-,-\}\) corresponds to the Schouten--Nijenhuis bracket \([-,-]\) on multivectors.
	\item A bivector \(\pi \in \Lambda^{2} \big(TM\big)\) corresponds to a quadratic function \(S_{\pi}\).
	\item The condition \(\{S_{\pi},S_{\pi}\} = 0\) is equivalent to \([\pi,\pi] = 0\), i.e. \(\pi\) is Poisson.
	\item The homological vector field \(Q \defeq \{S_{\pi},-\}\) satisfies \(Q^{2} = 0\) and is a derivation of the product.
	\item The derived bracket \(\{-,-\}_Q \defeq \{\{Q,-\},-\}\) recovers the Poisson bracket on superfunctions of degree 0.
\end{itemize}
\subsection{Supergeometric tangent vs.\ cotangent derived brackets}

\paragraph{Cotangent side: \(T^{*}[1]M\) (degree-1 symplectic NQ-manifold).}
Choose local coordinates \(x^{i}\) on \(M\) and fiber coordinates \(\xi_{i}\) of degree \(1\) on \(T^{*}[1]M\).
The canonical symplectic form of degree \(1\) is \(\omega = dx^{i}\wedge d\xi_{i}\).
It induces the graded Poisson bracket of degree \(-1\) with generators \(\{x^{i},\xi_{j}\}=\delta^{i}_{j}\), \(\{x^{i},x^{j}\}=0\), \(\{\xi_{i},\xi_{j}\}=0\).
Via the identification \(C^{\infty}(T^{*}[1]M)\cong\Gamma(\Lambda^{\bullet}TM)\), this bracket is the Schouten–Nijenhuis bracket.
A bivector \(\pi=\tfrac12\,\pi^{ij}(x)\,\partial_{i}\wedge\partial_{j}\) corresponds to the quadratic Hamiltonian \(S_{\pi}=\tfrac12\,\pi^{ij}(x)\,\xi_{i}\xi_{j}\).
The Poisson condition \([\pi,\pi]_{SN}=0\) is equivalent to the classical master equation \(\{S_{\pi},S_{\pi}\}=0\).
The homological vector field is \(Q_{\pi}\coloneqq\{S_{\pi},-\}\) and satisfies \(Q_{\pi}^{2}=0\).
For \(f,g\in C^{\infty}(M)\) (degree \(0\)), the \emph{derived bracket}
\[
\{f,g\}_{\pi}\;\coloneqq\;\big\{\{S_{\pi},f\},\,g\big\}
\]
agrees with the usual Poisson bracket \(\pi(df,dg)\).

\paragraph{Tangent side: \(T[1]M\) (Q-manifold and operator calculus).}
Coordinates \((x^{i},\theta^{i})\) have degrees \(|x^{i}|=0\), \(|\theta^{i}|=1\), and \(C^{\infty}(T[1]M)\cong\Omega^{\bullet}(M)\).
The canonical homological vector field is
\[
Q_{d}\;=\;\theta^{i}\,\frac{\partial}{\partial x^{i}},
\qquad
Q_{d}^{2}=0,
\]
which acts as the de Rham differential on forms.
For a vector field \(X=X^{i}(x)\partial_{i}\), the contraction operator is the vertical differential operator
\[
\hat\iota_{X}\;\coloneqq\;X^{i}(x)\,\frac{\partial}{\partial\theta^{i}},
\qquad
|\hat\iota_{X}|=-1,
\]
and for a homogeneous form \(\alpha\in\Omega^{k}(M)\), left wedge is \(m_{\alpha}(\beta)\coloneqq\alpha\wedge\beta\) with \(|m_{\alpha}|=k\).
Write the graded commutator on \(\End(\Omega^{\bullet}(M))\) as \([A,B]\coloneqq A\!\circ\!B-(-1)^{|A||B|}B\!\circ\!A\).
Cartan identities become
\[
[Q_{d},\hat\iota_{X}]=\widehat{\mathcal L_{X}},
\qquad
[\widehat{\mathcal L_{X}},\hat\iota_{Y}]=\hat\iota_{[X,Y]},
\qquad
[\hat\iota_{X},\hat\iota_{Y}]=0\ \text{(graded)}.
\]

\paragraph{Derived brackets on \(T[1]M\).}
\emph{Vector fields.}
Since the subalgebra \(\{\hat\iota_{X}\}\) is abelian for the graded commutator, the Lie bracket is a derived bracket:
\[
\big[\,[Q_{d},\hat\iota_{X}],\,\hat\iota_{Y}\,\big]\;=\;\hat\iota_{[X,Y]}.
\]
\emph{Poisson data on the tangent side.}
Let \(\pi=\tfrac12\,\pi^{ij}(x)\,\partial_{i}\wedge\partial_{j}\).
Contraction by \(\pi\) is the degree \(-2\) operator
\[
\hat\iota_{\pi}\;=\;\tfrac12\,\pi^{ij}(x)\,\frac{\partial}{\partial\theta^{j}}\frac{\partial}{\partial\theta^{i}}.
\]
Define the Koszul–Brylinski operator
\[
D_{\pi}\;\coloneqq\;[Q_{d},\hat\iota_{\pi}]\;=\;d\,\iota_{\pi}+\iota_{\pi}\,d,
\qquad
|D_{\pi}|=-1.
\]
If \(\pi\) is Poisson, then \(D_{\pi}^{2}=0\).
For homogeneous \(\alpha,\beta\in\Omega^{\bullet}(M)\), define the \emph{derived bracket on forms} by
\[
\{\alpha,\beta\}_{\pi} \defeq ...
\]
Then \(\deg\{\alpha,\beta\}_{\pi}=\deg\alpha+\deg\beta\), the operation is a graded Leibniz bracket, it vanishes on \(\Omega^{0}\), and on \(1\)-forms it is the Koszul bracket:
\[
[\alpha,\beta]_{\pi}
\;=\;
\mathcal L_{\pi^{\sharp}\alpha}\beta
-
\mathcal L_{\pi^{\sharp}\beta}\alpha
-
d\big(\pi(\alpha,\beta)\big)
\;=\;
\iota_{\pi^{\sharp}\alpha}d\beta
-
\iota_{\pi^{\sharp}\beta}d\alpha
+
d\big(\pi(\alpha,\beta)\big),
\]
with \(\pi^{\sharp}\colon T^{*}M\to TM\) given in coordinates by \((\pi^{\sharp}\alpha)^{i}=\pi^{ij}\alpha_{j}\).

\paragraph{Bridge between the two pictures.}
The map \(\phi_{\pi}\colon T^{*}[1]M\to T[1]M\), \((x,\xi)\mapsto(x,\theta=\pi^{\sharp}\xi)\), intertwines the \(Q\)-structures precisely when \(\pi\) is Poisson.
This encapsulates the compatibility of the cotangent \(Q_{\pi}=\{S_{\pi},-\}\) and the tangent \(Q_{d}\), and explains the agreement of the corresponding derived-bracket constructions.

