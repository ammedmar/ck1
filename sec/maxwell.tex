\subsection{Yang--Mills (Second order)}

\anibal{Notice the notion of Gauge fixing in CostelloGwilliams 2. It is an operator \( Q^{gf} \) whose commutatro with \( Q \) is a generalized laplacian. Pretty good for framing the BV box condition.}

\anibal{Another important point made in Costello-Gwilliams is that the degree -3 pairing in the L-infinity algerbra associated to a formal moduli problem correpsonds to a shifted symplectic structure on the moduli problem itself. They credit Kontsevich 93 for noticing this connection (at least in the noncommutative context.)}

\anibal{Formulas for the L-infty structure on the product of a Lie alg and a C-infty algebra are given in (6.7b) of Double copy from homotopy algebras}

\anibal{The symmetry of m3 is not stated or proven in the original Zeitlin paper, make sure to mention it and maybe prove it (works when start is even degree)}

\anibal{Careful that the Poisson relation is not a commutator as stated. It might have been used below, check.}

\anibal{Called second order Yang–Mills complex in L-infty of classical field theory and the BV formalism (5.62a)}

\anibal{Another important point: To get the YM equations in Zeitlin's Maurer-Cartan claim, we need to interpret [A,A] as 2AA, so the symmetry idea is already there, i.e. better suited for A infinity not C infinity}

In \cite{Zeitlin}, the \defn{Maxwell complex}
\[
\begin{tikzcd}
	0
	\rar &[-10pt] \Omega^0(M) \arrow[r, "d"] & \Omega^1(M) \arrow[r, "d \star d"] & \Omega^{D-1}(M) \arrow[r, "d"] & \Omega^D(M)
	\rar &[-10pt] 0
\end{tikzcd}
\]
of a Riemannian \( D \)-manifold \( M \) is consider, where \( d \) denotes the de Rham differential and \( \star \) the Hodge star.
Let us denote the above differential by \( m^1_1 \).
Zeitlin extends this complex to a \( \rC_{(2)} \)-algebra as follows:
\[
m^0_2(\alpha \ot \beta) =
\begin{cases}
	d\star(\alpha \wedge \beta) + \alpha \wedge (\star d\beta) - \beta \wedge  (\star d\alpha), &
	\bars{\alpha} = \bars{\beta} = 1, \\
	\hfil \alpha \wedge \beta, & \text{ otherwise}.
\end{cases}
\]
\[
m^0_3(\alpha \ot \beta \ot \gamma) =
\begin{cases}
	\alpha \wedge \star(\beta \wedge \gamma) + \star(\alpha \wedge \beta) \wedge \gamma, &
	\bars{\alpha} = \bars{\beta} = \bars{\gamma} = 1, \\
	\hfil 0 & \text{ otherwise}.\\
\end{cases}
\]
\anibal{I don't know if he checked \( \sM^0_4 \).}

\medskip We now extend non-trivially this structure to a \( \bvbox_\infty \)-algebra structure.
If \( d^\star \) is the Hodge codifferential, let \( m^1_1 \) be defined by
\[
\begin{tikzcd}[column sep=18pt]
	0
	&[-5pt] \Omega^0(M) \arrow[l] & \Omega^1(M) \arrow[l, "\ d^\star"'] & \Omega^{D-1}(M) \arrow[l, "\; \star"'] & \Omega^D(M)
	\arrow[l, "\ d^\star"'] &[-5pt] 0. \arrow[l]
\end{tikzcd}
\]
The obstruction \( n^1_1 = [m^0_1, m^1_1]\) is the Laplace--Beltrami operator.\anibal{Chech sign}.

The generating map \( m^1_1 \) does not square to zero.
In fact, since
%with Riemannina signature,
\[
\star \star \alpha = (-1)^{\bars{\alpha}(D-\bars{\alpha})} \, \alpha,
\]
one has
\[
(m^1_1 \circ m^1_1)(\alpha) =
\begin{cases}
	d\mathord{\star} \alpha& \bars{\alpha} = D, \\
	\star d \alpha & \bars{\alpha} = D-1, \\
	\hfil 0 & \text{ Otherwise}.
\end{cases}
\]
Defining
\[
m^2_1(\alpha) =
\begin{cases}
	- \star \alpha & \bars{\alpha} = D-1, D, \\
	\hfil 0 & \text{Otherwise},
\end{cases}
\]
we have \( n^2_1 = [m^0_1, m^2_1] + m^1_1 \circ m^1_1 = 0\).\anibal{Careful that \( d^\star \) is \( \pm \star d \star \)}

\medskip
Let us define \( m^0_{1,1} = [m^1_1, m^0_2]\) and \( m^1_2 = 0 \) so the obstruction \( n^1_2 \) vanishes.

Let us now compute the obstruction
\[
n^0_{1,1} = [m^0_1, m^0_{1,1}] = \big[m^0_1, [m^1_1, m^0_2]\big] = [n^1_1, m^0_2],
\]
to be the failure of the Laplace--Beltrami operator to be of second order with respect to the product.

Let us consider
\[
[m^0_2, m^0_{1,1}]
= \big[m^0_2, [m^1_1, m^0_2]\big]
= -\tfrac{1}{2}\big[m^1_1, [m^0_2, m^0_2]\big]
= -\big[m^1_1, [m^0_1, m^0_3]\big].
\]
Setting \( m^0_{1,2} = 0\) the obstruction \( n^0_{1,2} \) is \( \big[m^1_1, [m^0_1, m^0_3]\big] \).

Let us consider
\begin{align*}
	[m^1_1, m^0_{1,1}]
	&= \big[m^1_1, [m^1_1, m^0_2]\big]
	= -\tfrac{1}{2} \big[m^0_2, [m^1_1, m^1_1]\big] = - [m^0_2, (m^1_1 \circ m^1_1)] \\
	&= \big[m^0_2, [m^0_1, m^2_1]\big]
	= \big[m^0_1, [m^2_1, m^0_2]\big]
\end{align*}
Setting \( m^1_{1,1} = [m^2_1, m^0_2] \) we can let the obstruction \( n^1_{1,1} \) vanish.

Jacobi is not clearly split into boundary and obstruction yet...