\anibal{Learn more about Hochschild-Kostant-Rosenberg theorem as a way of describing BV in polyvector fields.}
\anibal{About super-geometry and the BV formalism: forms as functions on shifted tangent bundle.}

\subsection{Generating maps}\label{ss:generating_maps}

Let us fix a graded vector space \( A \).
A set of \defn{generating maps} consists of a linear map
\begin{align*}
	m_{p_1,\dots,p_k}^t &\colon
	A^{\ot p_1} \ot \dotsb \ot A^{\ot p_k}
	\longrightarrow A,
\end{align*}
for each \(	t \geqslant 0, \  k \geqslant 1, \  p_1, \dots, p_k \geqslant 1 \)\,, of degree
\[
\bars{m_{p_1,\dots,p_k}^t}  =  p_1 + \cdots + p_k + k + 2t - 3,
\]
satisfying the following \defn{block} and \defn{shuffle symmetries}:

\medskip\noindent(1)
For any permutation \( \sigma \in \sym_k \):
\[
m_{p_{\sigma(1)},\dots,p_{\sigma(k)}}^t \!= m_{p_1, \dots, p_k}^t \! \circ \overline\sigma,
\]
where \( \overline\sigma \) is the image of \( \sigma \) via the block inclusion \( \sym_k \to \sym_{p_1+\dots+p_k} \).

\medskip\noindent(2)
For any \( i \in \set{1,\dots,k}\) and \( j \in \set{1, \dots, p_i-1} \):
\begin{align*}
	&\sum_{\mathclap{\sigma \in \Sh(j,\, p_i - j)}} \ \sign(\sigma) \;
	m_{p_1,\dots,p_k}^t  \! \circ \underline\sigma =  0,
\end{align*}
where \( \underline\sigma \) is the image of \( \sigma \) in via the inclusion \( \sym_{p_i} \to \sym_{p_1+\dots+p_k} \) induced by the \( i^\th \)-block.

\subsection{Straight shuffle extensions}

To define the relations we will need the concept of \textit{straight shuffle}, which we illustrate with the following example:
\begin{equation}\label{ex:straight_shuffle}
	1\;\underline{2}\;3\;4\;|\;\underline{5\;6}\;7\;8\;|\;9\;\underline{10\;11}\;|\;\underline{12\;13}
	\ \longrightarrow\
	9\;1\;\;\underline{2\;5\;6\;10\;11\;12\;13}\;\;7\;3\;8\;4.
\end{equation}

\medskip\noindent Forgetting all decoration in \eqref{ex:straight_shuffle} we have the permutation
\[
\sigma =
\left[
\begin{array}{ccccccccccccc}
	1&2&3&4&5&6&7&8&9&10&11&12&13\\
	2&3&11&13&4&5&10&12&1&6&7&8&9
\end{array}
\right]
\in \sym_{13}.
\]
The segmentation induced by the vertical bars in \eqref{ex:straight_shuffle} is encoded in the \( k \)-tuple
\[
\bar p = (4,4,3,2)
\]
with \( k = 4 \).
The subdivision of each segment of \eqref{ex:straight_shuffle} induced by the underline is encoded in the three \( k \)-tuples
\[
\bar l = (1,0,1,0), \quad
\bar q = (1,2,2,2), \quad
\bar r = (2,2,0,0).
\]
By definition, \( \sigma \) is a \defn{ \( (\bar l, \bar q, \bar r, \bar p) \)-shuffle} if for all \( 1 \leqslant i \leqslant k \),
\[
(p_1+\dots+p_{i-1} + l_i) + [1, q_i ] \xmapsto{\sigma}
(l_1 + \dotsb + l_k) + (q_1 + \dotsb + q_{i-1}) + [1, q_i].
\]
Explicitly,
\[
2 \xmapsto{\sigma} 3, \quad
[5, 6] \xmapsto{\sigma} [4, 5], \quad
[10, 11] \xmapsto{\sigma} [6, 7], \quad
[12, 13] \xmapsto{\sigma} [8, 9].
\]
%Notice that \( \bar p = \bar l + \bar q + \bar r \) and that any other \( (\bar l, \bar q, \bar r, \bar p) \)-shuffle is of the form \( \underline\pi \circ \sigma \), where
%\[
%\begin{tikzcd}[row sep=0,column sep=small]
%	\sym_2 \times \sym_4 \rar & \sym_{2+7+4} \\
%	\pi \arrow[r, maps to] & \underline{\pi}~.
%\end{tikzcd}
%\]

We will be interested in the set  \( \mathrm{StSh}(\bar q, \bar p) \) of all straight shuffles for a fixed pair
\[
\bar q = (q_1, \dots, q_k) \leqslant (p_1, \dots, p_k) = \bar p.
\]

Let us now introduce some notation for the action of straight shuffles on tensor product elements of a graded vector space \( A \).
Using \( \sigma \), from example \eqref{ex:straight_shuffle}, we have that if \( x = x_1 \ot\dotsb\ot x_{13} \) then
\[
\sigma x =
\pm\, \overbrace{x_9 \ot x_1}^{\sigma x^{(1)}} \ot
\overbrace{x_2 \ot x_5 \ot x_6 \ot x_{10} \ot x_{11} \ot x_{12} \ot x_{13}}^{\sigma x^{(2)}} \ot
\overbrace{x_7 \ot x_3 \ot x_8 \ot x_4}^{\sigma x^{(3)}},
\]
where the sign is given by the permutation of elements under the Koszul sign rule.

\medskip As before, consider \( (q_1, \dots, q_k) \leqslant (p_1, \dots, p_k) \).
The \defn{straight shuffle extension} of a map
\[
m_{q_1, \dots, q_k} \colon A^{\ot q_1} \ot \dotsb \ot A^{\ot q_k} \to A
\]
to a map
\[
m_{\frac{p_1,\dots,p_k}{q_1,\dots,q_k}} \colon {A^{\ot p_1}} \ot\dotsb\ot {A^{\ot p_k}} \to A
\]
is defined by
\[
m_{\frac{p_1,\dots,p_k}{q_1,\dots,q_k}}(x)
=
\sum_{\quad \mathclap{\sigma \in \mathrm{StSh}(\bar q, \bar p)}}
%(-1)^{\varepsilon+\varepsilon'+\varepsilon''} \mathrm{sgn}(\sigma)\,
\, \pm \,
\sigma x^{(1)} \ot
m_{q_1,\dots,q_k}\big( \sigma x^{(2)} \big) \ot
\sigma x^{(3)},
\]
where the sign is given by the product of the following contributions:
the sign of \( \sigma \);
the Koszul sign arising from the permutation of \( \sigma x^{(1)} \) and \( m_{q_1,\dots,q_k} \);
and the explicit factor
\[
(-1)^{(q_1+\cdots+q_r+1)(r_1+\cdots+r_k) + (l_2+r_2) + 2(l_3+r_3) + \dotsb + (k-1)(l_k+r_k)}
\]
expressed under the assumption that  \( \sigma \) is a \( (\bar l, \bar q, \bar r, \bar p) \)-shuffle.

Notice that the Koszul sign associated to the action of \( \sigma \) on \( x \) is already incorporated in the notation \( \sigma x^{(1)} \ot \sigma x^{(2)} \ot \sigma x^{(3)} \).

\subsection{Relation maps}

%To describe the relations satisfied by the generating maps, we introduce the following \defn{relation maps}.
For each \( t \geqslant 0, \  k \geqslant 1, \  p_1, \dots, p_k \geqslant 1 \), the \defn{relation map} \( \sM^t_{p_1, \dots, p_k} \) is defined on the basis element
\[
\underbrace{(x_1 \ot\dotsb\ot x_{p_1})}_{w_1}
\ot \underbrace{(x_{P_1 + 1} \ot\dotsb\ot x_{P_1 + p_2})}_{w_2}
\ot\dotsb\ot
\underbrace{(x_{P_{k-1} + 1} \ot\dotsb\ot x_{P_k})}_{w_k}
\]
as the sum over \( 1 \leqslant s \leqslant t \) of
\begin{equation}\label{eq:relation_maps_1}
	\quad
	\sum_{\hspace*{10pt}\mathclap{\begin{array}{l}
				\scriptstyle I \,\sqcup J = \{1, \dots, k\} \\[-2pt]
				\scriptstyle I = \{i_1, \dots, i_a\} \neq \emptyset \\[-2pt]
				\scriptstyle J = \{j_1, \dots, j_b\}
	\end{array}}}
	\hspace*{35pt}  % space
	\sum_{\hspace*{0pt}\mathclap{\begin{array}{c}
				\scriptstyle 1 \leqslant q_1 \leqslant p_1 \\[-2pt]
				\scalebox{0.5}{\(\vdots\)} \\[-2pt]
				\scriptstyle 1 \leqslant q_a \leqslant p_a
	\end{array}}}
	\hspace*{3pt}	% space
	\pm \ m^{s}_{r,\,  p_{j_1}, \dots, p_{j_b}}
	\Big(m^{t-s}_{\frac{p_1, \dots, p_{i_a}}{q_1, \dots, q_a}}
	(w_{i_1} \!\ot\dotsb\ot w_{i_a}) \ot w_{j_1} \!\ot\dotsb\ot w_{j_b}\Big),
\end{equation}
where \( r = 1 + p_{i_1} + \dots + p_{i_a} - q_1 - \dots - q_a \)\,; minus the sum
\begin{equation}\label{eq:relation_maps_2}
	\sum_{1 \leq i \leq k}
	\hspace*{15pt}
	\sum_{\mathclap{1 \leqslant j \leqslant p_i-1}}
	\ \pm \, m^{t-1}_{p_1, \dots, j, p_i-j, \dots, p_k}
	\Big(w_1 \ot\dotsb\ot w_i^{(1)} \ot w_i^{(2)} \ot\dotsb\ot w_k\Big),
\end{equation}
where \( w_i^{(1)} = (x_{P_i+1} \ot\dotsb\ot x_{P_i+j}) \) and \( w_i^{(2)} = (x_{P_i+j+1} \ot\dotsb\ot x_{P_{i+1}})\).

\subsection{Signs}

The sign in \eqref{eq:relation_maps_1} is given by the product of the following three contributions:
the Koszul sign arising from the permutation of the elements, the sign of the \((a,b)\)-shuffle corresponding to the partition \(I \sqcup J = \{1, \dots, k\}\),
\anibal{From Bruno: Unfortunately, the situation is slightly more complicated: one should consider the blocks of size \( p_1-1, …, p_k-1 \) and shuffle them according to the permutation given by the partition \( I \sqcup J \). This implies for instance that if all the \( p_i=1 \), there is no sign. Is this clear? }
and the explicit factor
\[
(-1)^{(q_1+\dotsb+q_a - a + 1)\,(p_{j_1}+\cdots+p_{j_b} - b) + r - 1}.
\]
The sign in \eqref{eq:relation_maps_2} is equal to
\[
(-1)^{p_1+\cdots+p_{i-1}+j-i}.
\]

\subsection{Statements}

\begin{theorem}
	A set of generating maps \( \set{m_{p_1,\dots,p_k}^t} \) defines a \( \bvbox_\infty \)-algebra structure if \( \sM_p^0\) is identically \( 0 \) for every \( p \geq 1 \).
	Furtheremore, if in addition every \( \sM^t_{p_1, \dots, p_k} \) for \( k > 1 \) or \( t > 0 \) is identically \( 0 \), it defines a \( \bv_\infty \) algebra.
\end{theorem}

In a \( \bvbox_\infty \)-algebra, we refer to a relation map  \( \sM^t_{p_1, \dots, p_k} \) for \( k > 1 \) or \( t > 0 \) as an \defn{obstruction map} and denote it \( n^t_{p_1, \dots, p_k} \) for clarity.

\begin{theorem}
	RELATIONS
\end{theorem}

%
%\begin{theorem}[\cite{GalvezTonksVallette}]
%	A set of generating maps \( \set{m_{p_1,\dots,p_k}^t} \) defines a \( \bv_\infty \)-algebra structure if every \( \sM_{p_1,\dots,p_k}^t\) is identically \( 0 \).
%\end{theorem}
%
%\begin{corollary}
%	A set of generating maps \( \set{m_{p_1,\dots,p_k}^t} \) defines a \( \bvbox_\infty \)-algebra structure if \( \sM_p^0\) is identically \( 0 \) for every \( p \geq 1 \).
%\end{corollary}

