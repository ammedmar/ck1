% !TEX root = ../ck_dual.tex

\section{Bi-differential complexes}

\subsection{Conventions}

We work over a field of characteristic \( 0 \) and refer to \( \Z \)-graded vector spaces as \defn{complexes}.
The term \defn{map} is reserved for a linear map between these.
We say that a map \( f \colon V \to W \) is of \defn{degree} \( k \) if \( f(V_n) \subseteq W_{n+k} \) for all \( n \in \Z \).

A \defn{differential} \( d \colon V \to V \) is a nilpotent (i.e. \( d \circ d = 0 \)) map of degree \( 1 \).
We refer to the pair \( (V, d) \) as a \defn{differential complex}.
Similarly, a \defn{co-differential} \( d^\star \colon V \to V \) is a nilpotent map of degree \( -1 \) and the pair \( (V, d^\star) \) is referred to as  a \defn{co-differential complex}.

A \defn{morphism} of differential complexes is a degree 0 map \( f \colon V \to W \) commuting with the differentials (i.e., \( f \circ d_V = d_W \circ f\)).
Morphisms of co-differential complexes are defined analogously.

\subsection{Bi-differential complexes}

A \defn{bi-differential complex} is a triple \( (V, d, d^\star) \) with \( (V, d) \) a differential complex and \( (V, d^\star) \) a co-differential complex.

A \defn{morphism} of bi-differential complexes is a degree \( 0 \) map commuting with differentials and co-differentials.

An important example of an automorphism of bi-differential complexes is the \defn{box operator}:
\[
\square \defeq [d, d^\star] = [d^\star, d] = d \circ d^\star + d^\star \circ d
\]

A closely related notion is that of \textit{mixed complexes}; in our terminology, these are bi-differential complexes for which the box operator vanishes identically.

\begin{lemma}[Splitting]
	Any surjective morphism \( f \colon V \to W \) between bi-differential complexes with either of them bounded below or above has a section, i.e., a morphism \( g \colon W \to V \) such that \( f \circ g = \id_W \).
\end{lemma}

\begin{proof}
	Since any surjective linear map between vector spaces has a section, we can start with a degree \( 0 \) map \( s \colon W \to V \) which we will modify to be compatible with the bi-differential structure.
	For concreteness, consider the bounded below case.
	We will proceed by induction.
	...
\end{proof}

\subsection{Geometric examples}\label{ss:geometric examples}

A common source of bi-differential complexes is chain complexes together with a degree reversing idempotent automorphism.
Explicitly, if  \( \star_n \colon V_n \to V_{d-n}\) satisfies \( \star \circ \star = \id \) then \( d^\star \defeq \star d \star \) makes \( (V, d, d^\star) \) into a bi-differential complex.

\begin{example}
	(pseudo-)riemannian
\end{example}

\begin{example}
	Dolbeault
\end{example}

\begin{example}
	symplectic
\end{example}

\subsection{Contractions}

In the previous subsection, we presented several examples of bi-differential complexes arising naturally in geometry.
In this section, we describe how bi-differential complexes also emerge from deformation retractions.

Deformation retractions play a central role in Homological Perturbation Theory (HPT), where they provide a controlled setting for transferring algebraic structures between quasi-isomorphic complexes.
We will explore this further in the next section.

We now present deformation retractions and contractions in the context of differential complexes, while noting that an analogous treatment applies as well in the setting of co-differential complexes.

\medskip\noindent
A (strong) \defn{deformation retraction} consists of a pair of morphisms
\[
\begin{tikzcd}
	(V, d_V) \arrow[r, shift left=0.7ex, "p"] & (W, d_W) \arrow[l, shift left=0.7ex, "i"]
\end{tikzcd}
\]
together with a map \( h \colon V \to V \) of degree \( -1 \), called the \defn{contracting homotopy}, satisfying:
\[
p \circ i = \id_W, \qquad
i \circ p + d_V \circ h + h \circ d_V = \id_V.
\]

One can always modify the data of a deformation retraction so that it additionally satisfies the standard \defn{side conditions}:
\[
h \circ i = 0, \qquad
p \circ h = 0, \qquad
h \circ h = 0.\footnote{
\anibal{Maybe worth expanding on since we do something similar with the introduction of the operator \( G \).}
}
\]

In this case, the complex \( (W, d_W) \) is isomorphic to the subcomplex \( (\img(i \circ p), d_V) \).\footnote{
	Explicitly, consider the maps \( p \circ i \) and \( i \circ p|_{W'} \).
	By assumption, \( p \circ i = \id_W \), and for any \( w' = (i \circ p)(v) \in W' \), the identity \( h \circ i = 0 \) implies:
	\begin{align*}
		(i \circ p)(w')
		&= w' - (d \circ h + h \circ d)(w') \\
		&= w' - d \circ h \circ i \circ p(v) - h \circ d \circ i \circ p(v) \\
		&= w'.
	\end{align*}
}
This observation motivates the idea of ``internalizing'' the complex \( (W, d_W) \) into the data \( (V, d, h) \), leading to the notion of a \textit{contraction} introduced by Eilenberg--MacLane in \cite{??}.\anibal{change the notation for def retract to \( \pi, \iota \) reserving \( p \) as claimed}

In the language of this paper, a \defn{contraction} is a bi-differential complex \( (V, d, h) \) satisfying the identity:
\[
h \circ d \circ h = h.
\]
We define the associated \defn{projection} operator by
\[
p \defeq \id - d \circ h - h \circ d,
\]
and reserve the letter \( p \) for this operator throughout.
This terminology is justified because \( p \) is idempotent.\footnote{
Explicitly, we compute that
\begin{align*}
	p \circ p
	&= (\id - d \circ h - h \circ d) \circ (\id - d \circ h - h \circ d) \\
	&= \id - d \circ h - h \circ d \\
	&\quad - d \circ h + d \circ h \circ d \circ h + d \circ h \circ h \circ d \\
	&\quad - h \circ d + h \circ d \circ d \circ h + h \circ d \circ h \circ d \\ &=
	p.
\end{align*}
}
Therefore, \( V \) admits the direct sum decomposition:
\[
V = \img p \oplus \ker p.
\]

The following statement is a well-known fact that we state with a complete proof to make this presentation self-contained.

\begin{proposition}
	A contraction \( (V, d, h) \) determines a deformation retraction
	\begin{equation}\label{eq:def retract from contraction}
		\begin{tikzcd}
			\arrow[loop left, "h"] (V, d) \arrow[r, shift left=0.7ex, "p"]
			& (\img p, d) \arrow[l, shift left=0.7ex, "\ \id\, \oplus\, 0"]
		\end{tikzcd}
	\end{equation}
	that satisfies the side conditions.
	Conversely, a deformation retraction
	\[
	\begin{tikzcd}
		\arrow[loop left, "h"] (V, d_V) \arrow[r, shift left=0.7ex, "\pi"]
		& (W, d_W) \arrow[l, shift left=0.7ex, "\iota"]
	\end{tikzcd}
	\]
	satisfying the side conditions determines a contraction \( (V, d_V, h) \) whose projection agrees with \( (\iota \circ \pi) \) and \( (\img(\iota \circ \pi), d_V) \) is isomorphic to \( (W, d_W) \).

\end{proposition}

\begin{proof}
	Consider the contraction \( (V, d, h) \).
	That \eqref{eq:def retract from contraction} is a deformation retraction is immediate.
	The two non-trivial side conditions  are verified as follows:
	\begin{gather*}
		p \circ h = (\id - h \circ d - d \circ h) \circ h = h - h \circ d \circ h = 0, \\
		h \circ (\id_{\img p} \otimes 0) = h \circ p = h \circ (\id - h \circ d - d \circ h) = h - h \circ d \circ h = 0.
	\end{gather*}

	Conversely, consider a deformation retraction
	\[
	\begin{tikzcd}
		\arrow[loop left, "h"] (V, d_V) \arrow[r, shift left=0.7ex, "\pi"]
		& (W, d_W) \arrow[l, shift left=0.7ex, "\iota"]
	\end{tikzcd}
	\]
	satisfying the side conditions.
	We write \( d \) instead of \( d_V \) for the rest of this proof.
	To verify that \( h = h \circ d \circ h \) we simply apply \( h \) to the identity \( \iota \circ \pi = \id - d \circ h - h \circ d\) and use one of the side conditions.

	The canonical projection of the contraction is \( \id - d \circ h - h \circ d \) which is \( (\iota \circ \pi) \) by assumption.

	To verify that \( (\img(\iota \circ \pi), d) \) is isomorphic to \( (W, d_W) \) we consider the morphisms \( \iota \colon W \to V \), which satisfies \(  \img \iota \subseteq \img(\iota \circ \pi) \) since \( \pi \circ \iota = \id_W \), and the restriction of \( \pi \) to \( \img(\iota \circ \pi)\).
	We want to prove that they are inverse of each other.
	That \( \pi \circ \iota = \id_W \) holds by assumption.
	To obtain \( \iota \circ \pi|_{\img(\iota \circ \pi)} = \id_{\img(\iota \circ \pi)} \) we consider \( v = (\iota \circ \pi)(w) \) and use the idempotency of the canonical projection as follows:
	\[
	(\iota \circ \pi)(v) = (\iota \circ \pi \circ \iota \circ \pi) (w) = (p \circ p)(w) = p(w) = v.\qedhere
	\]\qedhere
\end{proof}

\subsection{Minimal contractions}

Homological Perturbation Theory permits the transfer of structure to any complex quasi-isomorphic to the original one.
Among these, homology plays a distinguished role, serving as the minimal and most canonical example.
We now identify the condition under which a contraction yields a deformation retraction onto the homology of the complex.

\begin{theorem}
	A contraction \( (V, d, h) \) defines a deformation retraction of \( (V, d) \) onto its homology if and only if
	\[
	d \circ h \circ d = d.
	\]
\end{theorem}

\begin{proof}
	Recall that a contraction \( (V, d, h) \) determines a deformation retraction onto \( (\img p, d) \), where \( p = \id - d \circ h - h \circ d \).
	In particular, \( (V, d) \) and \( (\img p, d) \) are quasi-isomorphic.
	Therefore, \( (\img p, d) \) is isomorphic to the homology \( \rH(V, d) \) if and only if \( d|_{\img p} = 0\).
	This condition is equivalent to the identity \( d \circ p = 0 \).
	Using the definition of \( p \), we compute:
	\[
	d \circ p
	= d - d \circ d \circ h - d \circ h \circ d
	= d - d \circ h \circ d.
	\]
	Therefore,
	\[
	d \circ p = 0 \quad \Leftrightarrow \quad d = d \circ h \circ d,
	\]
	which proves the claim.
\end{proof}

For a contraction \( (V, d, h) \), the condition \( d \circ h \circ d = d \), which characterizes \emph{minimal contractions}, complements the defining identity \( h \circ d \circ h = h \) and restores the symmetry inherent to bi-differential complexes.

\subsection{Bi-contractions}

In the next subsection, we investigate the contractions introduced above emphasizing their symmetric structure.

\begin{definition}
	A bi-differential complex \( (V, d, d^\star) \) is said to be a \defn{bi-contraction} if it satisfies:
	\[
	d^\star \circ d \circ d^\star = d^\star, \qquad
	d \circ d^\star \circ d = d.
	\]
\end{definition}

\begin{lemma}\label{thm:bi-contraction}
	Let \( (V, d, d^\star) \) be a bi-contraction.
	Then:
	\begin{gather*}
		\ker\square = \ker d \cap \ker d^\star, \\
		V = \ker\square \oplus \img d \oplus \img d^\star,
	\end{gather*}
	and \( \ker\square \) is canonically isomorphic to both \( \rH(V, d) \) and \( \rH(V, d^\star) \).
\end{lemma}

\begin{proof}
	Let \( p = \id - d \circ d^\star - d^\star \circ d \), and notice that by symmetry, this is the canonical projection for both contractions \( (V, d, d^\star) \) and \( (V, d^\star, d) \).

	\medskip\noindent
	\textbf{Claim 1:} \( \img p = \ker d \cap \ker d^\star \).

	\medskip\noindent (\( \subseteq \))
	Consider \( v = p v' = v' - d d^\star v' - d^\star d v' \).
	Applying \( d \) gives:
	\begin{align*}
		d v = d v' - d d d^\star v' - d d^\star d v' = d v' - d d^\star d v' = 0,
	\end{align*}
	using \( d^2 = 0 \).
	By symmetry, the same argument shows \( d^\star v = 0 \), hence \( v \in \ker d \cap \ker d^\star \).

	\medskip\noindent (\( \supseteq \))
	Consider \( v \in \ker d \cap \ker d^\star \).
	Applying \( p \) gives
	\[
	pv = v - d d^\star v - d^\star d v = v,
	\]
	so \( v \in \img p \).

	\medskip\noindent
	\textbf{Claim 2:} \( p \) induces isomorphisms
	\[
	\rH(V, d) \xra{H(p, d)} (\cH, 0) \xla{H(p, d^\star)} \rH(V, d^\star).
	\]
	Since \( (V, d, d^\star) \) is a contraction, \( p \) defines a quasi-isomorphism from \( (V, d) \) to \( (\img p, d|_{\img p}) \).
	But \( \img p = \cH = \ker d \cap \ker d^\star \), so \( d|_{\cH} = 0 \), and \( p \) induces an isomorphism from \( \rH(V, d) \) to \( \cH \).
	Similarly, the contraction \( (V, d^\star, d) \) yields an isomorphism \( \rH(V, d^\star) \cong \cH \).

	\medskip\noindent
	\textbf{Claim 3:} \( \ker p = \img d \oplus \img d^\star \).

	\medskip\noindent (\( \subseteq \))
	Consider \( v \in \ker p \).
	Then \( 0 = v - (d \circ d^\star)(v) - (d^\star \circ d)(v) \) or, equivalently, \( v =  d(d^\star v) + d^\star(dv)\) showing that \( v \in  \img d + \img d^\star\).

	\medskip\noindent (\( \supseteq \))
	For \( v \in \img d \), write \( v = d u \). Then
	\[
	p(d u) = d u - d d^\star d u - d^\star d d u = d u - d d^\star d u = 0,
	\]
	since \( d \circ d = 0 \) and \( d \circ d^\star \circ d = d \).
	Thus \( \img d \subseteq \ker p \), and similarly \( \img d^\star \subseteq \ker p \).

	\medskip\noindent (\( \oplus \))
	Let \( u \in \img d \cap \img d^\star \).
	Suppose \( u = d v = d^\star w \).
	Then \( d u = 0 \) and \( d^\star u = 0 \), so \( u \in \cH = \img p \).
	Since \( u \in \ker p \cap \img p \), we conclude \( u = 0 \), and the sum is direct.

	\medskip\noindent
	\textbf{Claim 4:} \( \ker d \cap \ker d^\star = \ker(d \circ d^\star + d^\star \circ d) \).

	\medskip\noindent (\( \subseteq \))
	If \( dx = d^\star x = 0 \) then \( (d \circ d^\star + d^\star \circ d)(x) = d(d^\star x) + d^\star(dx) = 0 \).

	\medskip\noindent (\( \subseteq \))
	If \( (d \circ d^\star + d^\star \circ d)(x) = 0 \) then \( dx = (d \circ d^\star \circ d)(x) = d \circ (d \circ d^\star + d^\star \circ d)(x) = 0 \).
	Similarly one computes that \( d^\star x = 0 \).
\end{proof}

\subsection{\( dd^\star\! \)-complexes}

The conditions defining a bi-contraction are quite restrictive, as none of the geometric examples of bi-differential complexes satisfy them.

\begin{comment}
	Revisiting these conditions using the box operator, we observe that a bi-differential complex \( (V, d, d^\star) \) is a bi-contraction if and only if \( \square = \id \) on \( (\img d + \img d^\star) \).\footnote{
	Explicitly, \( d = d \circ d^\star \circ d \) if and only if \( d = \square \circ d \), which holds if and only if \( \square|_{\img d} = \id \).
	A similar argument applies to the condition \( d^\star = d^\star \circ d \circ d^\star \), which is equivalent to \( \square|_{\img d^\star} = \id \).
}

Since \( \img\square \subseteq \img d + \img d^\star \) the bi-contraction property implies \( \img\square =  \img d + \img d^\star\).
As we will see, this weaker condition suffices to construct a \textit{Hodge-type decomposition} as the one constructed in \cref{thm:bi-contraction}.

\begin{definition}
	A \defn{\( dd^\star\! \)-complex} as a bi-differential complex \( (V, d, d^\star) \) satisfying the \defn{\( dd^\star\! \)-condition}:
	\[
	\img \square = \img d + \img d^\star.
	\]
\end{definition}

\begin{lemma}\label{thm:d dstar lemma}
	Let \( (V, d, d^\star) \) be a \( dd^\star\!\)-complex which is either bounded below or above.
	Then, there exists a bi-differential morphism \( G \colon \img\square \to V \) with \( \square \circ G = \id_{\img\square} \) such that \( (V, d, G \circ d^\star) \) and \( (V, G \circ d, d^\star) \) are bi-contractions.
\end{lemma}
\end{comment}

\begin{lemma}\label{thm:d dstar lemma}
	Let \( (V, d, d^\star) \) be a \( dd^\star\!\)-complex which is either bounded below or above.
	Then, there exists a bi-differential morphism \( G \colon \img\square \to V \) with \( \square \circ G = \id_{\img\square} \) such that \( (V, d, G \circ d^\star) \) and \( (V, G \circ d, d^\star) \) are bi-contractions.
\end{lemma}

\begin{proof}
	We use the splitting lemma for bi-differential complexes to construct \( G \).\anibal{Missing}

	\medskip\noindent
	We verify that \( (V, d, G \circ d^\star) \) satisfies the bi-contraction identities:
	\begin{align*}
		(G \circ d^\star) \circ d \circ (G \circ d^\star)
		&= G \circ (d^\star \circ d) \circ (G \circ d^\star) \\
		&= G \circ (\square - d \circ d^\star) \circ (G \circ d^\star) \\
		&= G \circ (\square \circ G) \circ d^\star - G \circ d \circ G \circ (d^\star \circ d^\star) \\
		&= G \circ d^\star,
	\end{align*}
	\begin{align*}
		d \circ (G \circ d^\star) \circ d
		&= G \circ (d \circ d^\star) \circ d \\
		&= G \circ (\square - d^\star \circ d) \circ d \\
		&= G \circ \square \circ d - G \circ d^\star \circ d \circ d \\
		&= d.
	\end{align*}
	The statement for \( (V, G \circ d, d^\star) \) follows by symmetry.
\end{proof}

\begin{lemma}\label{thm:d dstar hodge decomposition}
	Let \( (V, d, d^\star) \) be a \( dd^\star\!\)-complex that is bounded below or above.
	Then:
	\begin{gather*}
		\ker \square = \ker d \cap \ker d^\star, \\
		V = \ker \square \oplus \img d \oplus \img d^\star,
	\end{gather*}
	and \( \ker \square \) is canonically isomorphic to both \( \rH(V, d) \) and \( \rH(V, d^\star) \).
\end{lemma}

\begin{proof}
	Let \( G \colon \img\square \to V \) be a right inverse to the box operator \( \square = d d^\star + d^\star d \), as guaranteed by \cref{thm:d dstar lemma}.
	Since \( G \) is injective, we have:
	\[
	\ker(G \circ d) = \ker d, \qquad \ker(G \circ d^\star) = \ker d^\star.
	\]
	By \cref{thm:d dstar lemma}, the triples \( (V, d, G \circ d^\star) \) and \( (V, G \circ d, d^\star) \) are bi-contractions.
	Then, by \cref{thm:bi-contraction}, we obtain both, the identities:
	\begin{equation}\label{eq:d dstar hodge splitting}
		V = \ker \square \oplus \img d \oplus \img(G \circ d^\star) = \ker \square \oplus \img(G \circ d) \oplus \img d^\star,
	\end{equation}
	and the fact that \( \ker \square = \ker d \cap \ker d^\star \) is canonically isomorphic to both \( \rH(V, d) \) and \( \rH(V, d^\star) \).

	\smallskip\noindent
	To conclude, we claim that
	\[
	\img(G \circ d) = \img d \quad \text{and} \quad \img(G \circ d^\star) = \img d^\star.
	\]
	We verify the first identity; the second follows by symmetry.

	\medskip\noindent
	\( (\subseteq) \) Let \( G(dv) \in \img(G \circ d) \).
	By the decomposition \eqref{eq:d dstar hodge splitting}, we can write
	\[
	G(dv) = d u + G(d^\star w)
	\]
	for some \( u, w \in V \).
	Applying \( \square \) to both sides gives
	\[
	dv = \square(G(dv)) = \square(d u) + d^\star w.
	\]
	Then
	\[
	d(v - \square u) = d^\star w.
	\]
	This implies \( d^\star w \in \ker d \cap \ker(G \circ d^\star) \), so by the direct sum decomposition, \( G(d^\star w) = 0 \).
	Hence \( G(dv) = d u \in \img d \), proving \( \img(G \circ d) \subseteq \img d \).

	\medskip\noindent
	\( (\supseteq) \) Let \( du \in \img d \).
	Since \( (V, d, G \circ d^\star) \) is a bi-contraction, we have:
	\[
	du = (d \circ G \circ d^\star \circ d)(u) = (G \circ d \circ d^\star \circ d)(u) = (G \circ d)\big(d^\star(du)\big),
	\]
	so \( du \in \img(G \circ d) \), completing the proof.
\end{proof}