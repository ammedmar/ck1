% !TEX root = ../ck_dual.tex

\section{Bidifferential complexes}

\subsection{Conventions}

We work over a field \( \k \) of characteristic \( 0 \) and refer to \( \Z \)-graded vector spaces as \defn{complexes}.
The term \defn{map} is reserved for a linear map between these.
We say that a map \( f \colon V \to W \) is of \defn{degree} \( k \) if \( f(V_n) \subseteq W_{n+k} \) for all \( n \in \Z \).

A \defn{differential} \( d \colon V \to V \) is a nilpotent (i.e. \( d \circ d = 0 \)) map of degree \( 1 \).
We refer to the pair \( (V, d) \) as a \defn{differential complex}.
Similarly, a \defn{codifferential} \( d^\star \colon V \to V \) is a nilpotent map of degree \( -1 \) and the pair \( (V, d^\star) \) is referred to as  a \defn{codifferential complex}.

A \defn{morphism} of differential complexes is a degree 0 map \( f \colon V \to W \) commuting with the differentials (i.e., \( f \circ d_V = d_W \circ f\)).
Morphisms of codifferential complexes are defined analogously.

\subsection{Bidifferential complexes}

A \defn{bidifferential complex} is a triple \( (V, d, d^\star) \) where \( (V, d) \) is a differential complex and \( (V, d^\star) \) is a codifferential complex.

A \defn{morphism} of bidifferential complexes is a degree \( 0 \) linear map that commutes with both \( d \) and \( d^\star \).

Every bidifferential complex is equipped with a canonical endomorphism known as the \defn{box operator}:
\[
\square \defeq [d, d^\star] = [d^\star, d] = d \circ d^\star + d^\star \circ d.
\]

When \( \square = 0 \), the bidifferential complex \( (V, d, d^\star) \) is referred to in the literature as a \textit{mixed complex}.
This note, however, focuses on the case where \( \square \neq 0 \).

\subsection{Examples}\label{ss:geometric_examples}

%%A \defn{star complex} is a bounded differential complex
%\[
%V_0 \xra{d} V_1 \xra{d} \dotsb \xra{d} V_{n-1} \xra{d} V_n
%\]
%together with a linear automorphism
%\[
%\star_m \colon V_m \to V_{n-m}
%\]
%for every \( m \in \set{0, \dots, n} \) satisfying \( \star_{n-m} \circ \star_m = \epsilon_m \cdot \id_{V_m} \) for some invertible scalar \( \epsilon_m \).
%It defines canonically a bidifferential complex whose codifferential is \( d^\star \defeq \star \circ d \circ \star \).

We will now review some examples of bidifferential complexes arising in geometry and combinatorial topology.

\begin{example}
	Let \( (M, g) \) be an oriented pseudo-Riemannian manifold of dimension~\( n \) and \( (\Omega(M), d) \) denote its complex of differential forms.
	The \defn{Hodge star} operator
	\[
	\star \colon \Omega^m(M) \to \Omega^{n-m}(M)
	\]
	is the unique operator satisfying
	\[
	\alpha \wedge \star \beta = \langle \alpha, \beta \rangle_g \, \vol_g
	\]
	for all \( \alpha, \beta \in \Omega^m(M) \), where \( \vol_g \) is the oriented volume form on \( M \) associated to \( g \) and \( \langle -, - \rangle_g \) is the canonical extension of the metric to differential forms.

	This operator satisfies
	\[
	\star \circ \star = (-1)^{m(n-m) + s} \cdot \id \quad \text{on } \Omega^m(M),
	\]
	where \( s \) is the signature index of the metric (e.g., \( s = 0 \) for Riemannian, \( s = 1 \) for Lorentzian).

	The triple \( (\Omega(M), d, \star) \) is a star complex whose associated bidifferential complex has box operator \( \square = [d, d^\star] \) equal to the \defn{Laplace–Beltrami operator} if \( s = 0 \) or to the \defn{d'Alembertian} if \( s = 1 \).
\end{example}

\begin{example}
	Discrete laplacian...
\end{example}

\begin{example}
	Let \( (M, h) \) be a Hermitian manifold of complex dimension~\( n \), and \( (\Omega^{0,\bullet}(M), \bar\partial) \) its Dolbeault complex of antiholomorphic forms.

	Similar to the previous example, the
	%	Hermitian metric \( h \) induces a pointwise Hermitian inner product
	%	\[
	%	\langle \alpha, \beta \rangle_h \quad \text{on } \Omega^{0,m}(M),
	%	\]
	conjugate-linear \defn{Hodge star} operator
	\[
	\star \colon \Omega^{0,m}(M) \to \Omega^{0,n-m}(M)
	\]
	is defined via the identity
	\[
	\alpha \wedge \star \bar\beta = \langle \alpha, \beta \rangle_h \, \vol_h.
	\]

	This operator satisfies
	\[
	\star \circ \star = (-1)^m \cdot \id \quad \text{on } \Omega^{0,m}(M).
	\]

	The triple \( (\Omega^{0,\bullet}(M), \bar\partial, \star) \) is a star complex, and the induced bidifferential complex has codifferential \( \bar\partial^\star = \star \circ \bar\partial \circ \star \).
	The associated box operator \( \square = [\bar\partial, \bar\partial^\star] \) is the \defn{Dolbeault Laplacian}.
\end{example}

\begin{example}
	Let \( (M, \omega) \) be a symplectic manifold of dimension~\( 2n \), and \( (\Omega^\bullet(M), d) \) its complex of differential forms.

	The symplectic form \( \omega \) induces a canonical \defn{symplectic star} operator
	\[
	\star_\omega \colon \Omega^m(M) \to \Omega^{2n - m}(M)
	\]
	satisfying
	\[
	\alpha \wedge \star_\omega \beta = \langle \alpha, \beta \rangle_\omega \, \frac{\omega^n}{n!},
	\]
	where \( \langle -, - \rangle_\omega \) is the inner product on forms induced by the symplectic volume and symplectic duality.

	The operator satisfies
	\[
	\star_\omega \circ \star_\omega = \id \quad \text{on } \Omega^m(M),
	\]
	and defines a codifferential
	\[
	d^\star \defeq \star_\omega \circ d \circ \star_\omega.
	\]

	The triple \( (\Omega^\bullet(M), d, \star_\omega) \) is a star complex, and the associated bidifferential complex has box operator \( \square = [d, d^\star] \), which plays a central role in symplectic Hodge theory.
\end{example}

\subsection{Contractions}

A LOT OF PAPERS TO CITE HERE. MOST PRESSING ARE THE WORK OF CHUANG-LAZAREV AND STASHEFF ET AL.

In the previous subsection, we presented several examples of bidifferential complexes arising naturally in geometry.
In this section, we describe how bidifferential complexes also emerge from deformation retractions, which provide a controlled setting for transferring derived algebraic structures between quasi-isomorphic complexes.

We now present deformation retractions and contractions in the context of differential complexes, while noting that an analogous treatment applies as well in the setting of codifferential complexes.

\medskip\noindent
A (strong) \defn{deformation retraction} consists of a pair of morphisms
\[
\begin{tikzcd}
	(V, d_V) \arrow[r, shift left=0.7ex, "p"] & (W, d_W) \arrow[l, shift left=0.7ex, "i"]
\end{tikzcd}
\]
together with a map \( h \colon V \to V \) of degree \( -1 \), called the \defn{contracting homotopy}, satisfying:
\[
p \circ i = \id_W, \qquad
i \circ p + d_V \circ h + h \circ d_V = \id_V.
\]

One can always modify the data of a deformation retraction so that it additionally satisfies the standard \defn{side conditions}:
\[
h \circ i = 0, \qquad
p \circ h = 0, \qquad
h \circ h = 0.\footnote{
\anibal{ If \( h \circ h \neq 0 \) then let \( \tilde h \defeq d\)}
}
\]

In this case, the complex \( (W, d_W) \) is isomorphic to the subcomplex \( (\img(i \circ p), d_V) \).\footnote{
	Explicitly, consider the maps \( p \circ i \) and \( i \circ p|_{W'} \).
	By assumption, \( p \circ i = \id_W \), and for any \( w' = (i \circ p)(v) \in W' \), the identity \( h \circ i = 0 \) implies:
	\begin{align*}
		(i \circ p)(w')
		&= w' - (d \circ h + h \circ d)(w') \\
		&= w' - d \circ h \circ i \circ p(v) - h \circ d \circ i \circ p(v) \\
		&= w'.
	\end{align*}
}
This observation motivates the idea of ``internalizing'' the complex \( (W, d_W) \) into the data \( (V, d, h) \), leading to the notion of a \textit{contraction} introduced by Eilenberg--MacLane in \cite{??}.\anibal{change the notation for def retract to \( \pi, \iota \) reserving \( p \) as claimed}

In the language of this paper, a \defn{contraction} is a bidifferential complex \( (V, d, h) \) satisfying the identity:
\[
h \circ d \circ h = h.
\]
We define the associated \defn{projection} operator by
\[
p \defeq \id - d \circ h - h \circ d,
\]
and reserve the letter \( p \) for this operator throughout.
This terminology is justified because \( p \) is idempotent.\footnote{
Explicitly, we compute that
\begin{align*}
	p \circ p
	&= (\id - d \circ h - h \circ d) \circ (\id - d \circ h - h \circ d) \\
	&= \id - d \circ h - h \circ d \\
	&\quad - d \circ h + d \circ h \circ d \circ h + d \circ h \circ h \circ d \\
	&\quad - h \circ d + h \circ d \circ d \circ h + h \circ d \circ h \circ d \\ &=
	p.
\end{align*}
}
Therefore, \( V \) admits the direct sum decomposition:
\[
V = \img p \oplus \ker p.
\]

The following statement is a well-known fact that we write with a complete proof to make this presentation self-contained.

\begin{proposition}
	A contraction \( (V, d, h) \) determines a deformation retraction
	\begin{equation}\label{eq:def retract from contraction}
		\begin{tikzcd}
			\arrow[loop left, "h"] (V, d) \arrow[r, shift left=0.7ex, "p"]
			& (\img p, d) \arrow[l, shift left=0.7ex, "\ \id\, \oplus\, 0"]
		\end{tikzcd}
	\end{equation}
	that satisfies the side conditions.
	Conversely, a deformation retraction
	\[
	\begin{tikzcd}
		\arrow[loop left, "h"] (V, d_V) \arrow[r, shift left=0.7ex, "\pi"]
		& (W, d_W) \arrow[l, shift left=0.7ex, "\iota"]
	\end{tikzcd}
	\]
	satisfying the side conditions determines a contraction \( (V, d_V, h) \) whose projection agrees with \( (\iota \circ \pi) \) and \( (\img(\iota \circ \pi), d_V) \) is isomorphic to \( (W, d_W) \).

\end{proposition}

\begin{proof}
	Consider the contraction \( (V, d, h) \).
	That \eqref{eq:def retract from contraction} is a deformation retraction is immediate.
	The two non-trivial side conditions  are verified as follows:
	\begin{gather*}
		p \circ h = (\id - h \circ d - d \circ h) \circ h = h - h \circ d \circ h = 0, \\
		h \circ (\id_{\img p} \otimes 0) = h \circ p = h \circ (\id - h \circ d - d \circ h) = h - h \circ d \circ h = 0.
	\end{gather*}

	Conversely, consider a deformation retraction
	\[
	\begin{tikzcd}
		\arrow[loop left, "h"] (V, d_V) \arrow[r, shift left=0.7ex, "\pi"]
		& (W, d_W) \arrow[l, shift left=0.7ex, "\iota"]
	\end{tikzcd}
	\]
	satisfying the side conditions.
	We write \( d \) instead of \( d_V \) for the rest of this proof.
	To verify that \( h = h \circ d \circ h \) we simply apply \( h \) to the identity \( \iota \circ \pi = \id - d \circ h - h \circ d\) and use one of the side conditions.

	The canonical projection of the contraction is \( \id - d \circ h - h \circ d \) which is \( (\iota \circ \pi) \) by assumption.

	To verify that \( (\img(\iota \circ \pi), d) \) is isomorphic to \( (W, d_W) \) we consider the morphisms \( \iota \colon W \to V \), which satisfies \(  \img \iota \subseteq \img(\iota \circ \pi) \) since \( \pi \circ \iota = \id_W \), and the restriction of \( \pi \) to \( \img(\iota \circ \pi)\).
	We want to prove that they are inverse of each other.
	That \( \pi \circ \iota = \id_W \) holds by assumption.
	To obtain \( \iota \circ \pi|_{\img(\iota \circ \pi)} = \id_{\img(\iota \circ \pi)} \) we consider \( v = (\iota \circ \pi)(w) \) and use the idempotency of the canonical projection as follows:
	\[
	(\iota \circ \pi)(v) = (\iota \circ \pi \circ \iota \circ \pi) (w) = (p \circ p)(w) = p(w) = v.\qedhere
	\]\qedhere
\end{proof}

\subsection{Minimal contractions}

The homotopy transfer theorem ensures that algebraic structures defined on a chain complex can be transferred to any quasi-isomorphic complex.
Among these, homology plays a distinguished role, serving as the minimal and most canonical example.
We now identify the condition under which a contraction yields a deformation retraction onto the homology of the complex.

\begin{theorem}
	A contraction \( (V, d, h) \) defines a deformation retraction of \( (V, d) \) onto its homology if and only if
	\[
	d \circ h \circ d = d.
	\]
\end{theorem}

\begin{proof}
	Recall that a contraction \( (V, d, h) \) determines a deformation retraction onto \( (\img p, d) \), where \( p = \id - d \circ h - h \circ d \).
	In particular, \( (V, d) \) and \( (\img p, d) \) are quasi-isomorphic.
	Therefore, \( (\img p, d) \) is isomorphic to the homology \( \rH(V, d) \) if and only if \( d|_{\img p} = 0\).
	This condition is equivalent to the identity \( d \circ p = 0 \).
	Using the definition of \( p \), we compute:
	\[
	d \circ p
	= d - d \circ d \circ h - d \circ h \circ d
	= d - d \circ h \circ d.
	\]
	Therefore,
	\[
	d \circ p = 0 \quad \Leftrightarrow \quad d = d \circ h \circ d,
	\]
	which proves the claim.
\end{proof}

For a contraction \( (V, d, h) \), the condition \( d \circ h \circ d = d \), which characterizes \emph{minimal contractions}, complements the defining identity \( h \circ d \circ h = h \) and restores the symmetry inherent to bidifferential complexes.

\subsection{Bicontractions}

%In the next subsection, we investigate the contractions introduced above emphasizing their symmetric structure.

\begin{definition}
	A bidifferential complex \( (V, d, d^\star) \) is said to be a \defn{bicontraction} if it satisfies:
	\[
	d^\star \circ d \circ d^\star = d^\star
	\qquad\text{and}\qquad
	d \circ d^\star \circ d = d.
	\]
\end{definition}

\begin{lemma}\label{thm:bicontraction}
	Let \( (V, d, d^\star) \) be a bicontraction.
	Then:
	\begin{gather*}
		\ker\square = \ker d \cap \ker d^\star, \\
		V = \ker\square \oplus \img d \oplus \img d^\star,
	\end{gather*}
	and \( \ker\square \) is canonically isomorphic to both \( \rH(V, d) \) and \( \rH(V, d^\star) \).
\end{lemma}

\begin{proof}
	Let \( p = \id - d \circ d^\star - d^\star \circ d \), and notice that by symmetry, this is the canonical projection for both contractions \( (V, d, d^\star) \) and \( (V, d^\star, d) \).

	\medskip\noindent
	\textbf{Claim 1:} \( \img p = \ker d \cap \ker d^\star \).

	\medskip\noindent (\( \subseteq \))
	Consider \( v = p v' = v' - d d^\star v' - d^\star d v' \).
	Applying \( d \) gives:
	\begin{align*}
		d v = d v' - d d d^\star v' - d d^\star d v' = d v' - d d^\star d v' = 0,
	\end{align*}
	using \( d^2 = 0 \).
	By symmetry, the same argument shows \( d^\star v = 0 \), hence \( v \in \ker d \cap \ker d^\star \).

	\medskip\noindent (\( \supseteq \))
	Consider \( v \in \ker d \cap \ker d^\star \).
	Applying \( p \) gives
	\[
	pv = v - d d^\star v - d^\star d v = v,
	\]
	so \( v \in \img p \).

	\medskip\noindent
	\textbf{Claim 2:} \( p \) induces isomorphisms
	\[
	\rH(V, d) \xra{H(p, d)} (\cH, 0) \xla{H(p, d^\star)} \rH(V, d^\star).
	\]
	Since \( (V, d, d^\star) \) is a contraction, \( p \) defines a quasi-isomorphism from \( (V, d) \) to \( (\img p, d|_{\img p}) \).
	But \( \img p = \cH = \ker d \cap \ker d^\star \), so \( d|_{\cH} = 0 \), and \( p \) induces an isomorphism from \( \rH(V, d) \) to \( \cH \).
	Similarly, the contraction \( (V, d^\star, d) \) yields an isomorphism \( \rH(V, d^\star) \cong \cH \).

	\medskip\noindent
	\textbf{Claim 3:} \( \ker p = \img d \oplus \img d^\star \).

	\medskip\noindent (\( \subseteq \))
	Consider \( v \in \ker p \), i.e., \( 0 = v - (d \circ d^\star)(v) - (d^\star \circ d)(v) \) or, equivalently, \( v =  d(d^\star v) + d^\star(dv)\), showing that \( v \in  \img d + \img d^\star\).

	\medskip\noindent (\( \supseteq \))
	For \( v \in \img d \), write \( v = d u \). Then
	\[
	p(d u) = d u - d d^\star d u - d^\star d d u = d u - d d^\star d u = 0,
	\]
	since \( d \circ d = 0 \) and \( d \circ d^\star \circ d = d \).
	Thus \( \img d \subseteq \ker p \), and similarly \( \img d^\star \subseteq \ker p \).

	\medskip\noindent (\( \oplus \))
	Let \( u \in \img d \cap \img d^\star \).
	Suppose \( u = d v = d^\star w \).
	Then \( d u = 0 \) and \( d^\star u = 0 \), so \( u \in \cH = \img p \).
	Since \( u \in \ker p \cap \img p \), we conclude \( u = 0 \), and the sum is direct.

	\medskip\noindent
	\textbf{Claim 4:} \( \ker d \cap \ker d^\star = \ker(d \circ d^\star + d^\star \circ d) \).

	\medskip\noindent (\( \subseteq \))
	If \( dx = d^\star x = 0 \) then \( (d \circ d^\star + d^\star \circ d)(x) = d(d^\star x) + d^\star(dx) = 0 \).

	\medskip\noindent (\( \subseteq \))
	If \( (d \circ d^\star + d^\star \circ d)(x) = 0 \) then \( dx = (d \circ d^\star \circ d)(x) = d \circ (d \circ d^\star + d^\star \circ d)(x) = 0 \).
	Similarly one computes that \( d^\star x = 0 \).
\end{proof}

\subsection{Green operators}

%A right inverse for \( \square \) is referred to as a \emph{Green operator}.

Suppose \( V \) admits a direct sum decomposition
\[
V = \img(\square) \oplus \ker(\square),
\]
a condition satisfied, for example, when \( \square \) is an elliptic operator.
Let \( \pi \) denote the projection onto \( \ker(\square) \).
The operator \( \square + \pi \) is invertible, as its kernel is trivial and its image is all of \( V \).
In this case we have a canonical right inverse for \( \square \), given by
\begin{equation}\label{eq:Green_operator}
	G \defeq (\square + \pi)^{-1} \circ (\id_V - \pi).
\end{equation}

More generally, one may consider the existence of a projection \( \pi \colon V \to V \) onto a complement of \( \img(\square) \) such that the operator \( \square + \pi \) is invertible.
If this holds, the above construction \eqref{eq:Green_operator} applies verbatim and we say that the bidifferential complex \defn{admits a Green operator}, which we somewhat abusively denote \( G \) for any choice of such \( \pi \).

\begin{lemma}
	If \( (V, d, d^\star) \) admits a Green operator, then \( G \) restricts to a bidifferential automorphism of \( \img \square \).
\end{lemma}

\begin{proof}
	Consider a projection as provided by the assumption: \( \pi \colon V \to \img \pi \), with \( V = \img \square \oplus \img \pi \) and \( \square + \pi \) invertible.

	Since \( G = (\square + \pi)^{-1} \) on \( \img \square \), to prove that it is an automorphism of \( \img \square \), it suffices to show that its image is equal to \( \img \square \).
	Consider \( G(\square c) = \square c' + \pi c'' \) for some \( c, c', c'' \) and apply \( \square + \pi \) to this identity:
	\[
	\square c = \square(c' + \pi c'') + \pi c''.
	\]
	Since \( \square \) and \( \pi \) have complementary images, we conclude that \( \pi c'' = 0 \), as required.

	Let us now prove that \( G \) commutes with \( d \) and \( d^\star \) when restricted to \( \img \square \).
	For any \( c \in \img \square \), the element \( c' = (G \circ d)(c) \) is characterized by \( \square c' = d c \).
	The same identity is satisfied by \( c'' = (d \circ G)(c) \) since \( d \circ \square = \square \circ d \), so \( c' = c'' \), i.e., \( (G \circ d)(c) = (d \circ G)(c) \).
	The same argument applies to \( d^\star \).
\end{proof}

\subsection{\( dd^\star \)-complexes}

\begin{definition}
	A \defn{\( dd^\star\! \)-complex} is a bidifferential complex \( (V, d, d^\star) \) satisfying the \defn{\( dd^\star\! \)-condition}:
	\[
	\img \square = \img d + \img d^\star.
	\]
\end{definition}

\begin{theorem}\label{thm:d dstar lemma}
	If \( (V, d, d^\star) \) is a \( dd^\star \)-complex admitting a Green operator \( G \) then both \( (V, d, G \circ d^\star) \) and \( (V, G \circ d, d^\star) \) are bicontractions.
\end{theorem}

\begin{proof}
	We verify that \( (V, d, G \circ d^\star) \) satisfies the bicontraction identities:
	\begin{align*}
		(G \circ d^\star) \circ d \circ (G \circ d^\star)
		&= G \circ (d^\star \circ d) \circ (G \circ d^\star) \\
		&= G \circ (\square - d \circ d^\star) \circ (G \circ d^\star) \\
		&= G \circ (\square \circ G) \circ d^\star - G \circ d \circ G \circ (d^\star \circ d^\star) \\
		&= G \circ d^\star,
	\end{align*}
	\begin{align*}
		d \circ (G \circ d^\star) \circ d
		&= G \circ (d \circ d^\star) \circ d \\
		&= G \circ (\square - d^\star \circ d) \circ d \\
		&= G \circ \square \circ d - G \circ d^\star \circ d \circ d \\
		&= (\square + \pi)^{-1} \circ \square \circ d \\
		&= (\square + \pi)^{-1} \circ (\square + \pi) \circ d \\
		&= d.
	\end{align*}
	The statement for \( (V, G \circ d, d^\star) \) follows by symmetry.
\end{proof}


\begin{corollary}\label{thm:d dstar hodge decomposition}
	Let \( (V, d, d^\star) \) be a \( dd^\star \)-complex admitting a Green operator.
	Then:
	\begin{gather*}
		\ker \square = \ker d \cap \ker d^\star, \\
		V = \ker \square \oplus \img d \oplus \img d^\star,
	\end{gather*}
	and \( \ker \square \) is canonically isomorphic to both \( \rH(V, d) \) and \( \rH(V, d^\star) \).
\end{corollary}

\begin{proof}
	Let \( G \colon \img\square \to V \) be a Green operator, i.e., a right inverse to the box operator \( \square = d d^\star + d^\star d \).
	Since \( G \) is injective, we have:
	\[
	\ker(G \circ d) = \ker d, \qquad \ker(G \circ d^\star) = \ker d^\star.
	\]
	By \cref{thm:d dstar lemma}, the triples \( (V, d, G \circ d^\star) \) and \( (V, G \circ d, d^\star) \) are bicontractions.
	Then, by \cref{thm:bicontraction}, we obtain both, the identities:
	\begin{equation}\label{eq:d dstar hodge splitting}
		V = \ker \square \oplus \img d \oplus \img(G \circ d^\star) = \ker \square \oplus \img(G \circ d) \oplus \img d^\star,
	\end{equation}
	and the fact that \( \ker \square = \ker d \cap \ker d^\star \) is canonically isomorphic to both \( \rH(V, d) \) and \( \rH(V, d^\star) \).

	\smallskip\noindent
	To conclude, we claim that
	\[
	\img(G \circ d) = \img d \quad \text{and} \quad \img(G \circ d^\star) = \img d^\star.
	\]
	We verify the first identity; the second follows by symmetry.

	\medskip\noindent
	\( (\subseteq) \) Let \( G(dv) \in \img(G \circ d) \).
	By the decomposition \eqref{eq:d dstar hodge splitting}, we can write
	\[
	G(dv) = d u + G(d^\star w)
	\]
	for some \( u, w \in V \).
	Applying \( \square \) to both sides gives
	\[
	dv = \square(G(dv)) = \square(d u) + d^\star w.
	\]
	Then
	\[
	d(v - \square u) = d^\star w.
	\]
	This implies \( d^\star w \in \ker d \cap \ker(G \circ d^\star) \), so by the direct sum decomposition, \( G(d^\star w) = 0 \).
	Hence \( G(dv) = d u \in \img d \), proving \( \img(G \circ d) \subseteq \img d \).

	\medskip\noindent
	\( (\supseteq) \) Let \( du \in \img d \).
	Since \( (V, d, G \circ d^\star) \) is a bicontraction, we have:
	\[
	du = (d \circ G \circ d^\star \circ d)(u) = (G \circ d \circ d^\star \circ d)(u) = (G \circ d)\big(d^\star(du)\big),
	\]
	so \( du \in \img(G \circ d) \), completing the proof.
\end{proof}