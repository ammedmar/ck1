% !TEX root = ../ck_dual.tex

\section{Color-kinematic duality}

Color-kinematics duality is a surprising structure in perturbative quantum field theory, particularly evident in Yang--Mills theory.
At tree level, it states that scattering amplitudes can be reorganized so that the kinematic numerators (which encode dependence on momenta and polarizations) obey the same antisymmetry and Jacobi relations as the color numerators (built from the structure constants of a Lie algebra)

Mathematically, this duality is best understood through the lens of homotopy algebras.
The key insight is that both the color and kinematic structures can be encoded in a factorization of an underlying differential graded Lie algebra as a tensor product:
\[
\cL \cong \fg \otimes \cB
\]
where \( (\cB, d, d^\star, \wedge) \) is a \( dd^\star\! \)-algebra and the bracket of \( \cL \) is given by
\[
[g_1 \ot b_1, g_2 \ot b_2] = [g_1, g_2] \ot b_1 \wedge b_2.
\]

The scattering amplitudes associated to \( \cL \) are related to the \( L_\infty \)-structure induced on the minimal model \( \rH(\cL, d) \) of \( \cL \).
We can use the  bi-contraction \( (\cB, d, g \circ d^\star) \) associated to a choice of section \( \square \circ G = \id_{\img \square} \) discussed in \cref{thm:d dstar lemma}


That is, on the higher brackets induced on the \( \rH(\cL, d) \)
\[
\ell_k = \sum_{\gamma \in BT^r(k)} p \circ
\]

