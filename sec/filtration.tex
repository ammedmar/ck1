\subsection{Weight filtration}

We define the \defn{weight} of a generating map \( m_{p_1, \dots, p_k}^t \) to be \( t + p_1 + \dots + p_k - 1 \).

A \( \bv_\infty \)-algebra (resp.\ \( \bvbox_\infty \)-algebra) structure on \( A \) in which all generating maps of weight greater than \( n \) vanish is called a \defn{\( \bv_{\!(n)} \)-algebra} (resp.\ \defn{\( \bvbox_{\!(n)} \)-algebra}).

Defining similarly the weight of a relation map \( \sM_{p_1, \dots, p_k}^t \) to be \( t + p_1 + \dots + p_k - 1 \), we observe that in a \( \bv_{\!(n)} \)- or \( \bvbox_{\!(n)} \)-algebra, all relation maps of weight greater than \( w \) are linear combinations of identically zero maps.

Therefore, in order to construct a \( \bv_{\!(n)} \)- or \( \bvbox_{\!(n)} \)-algebra structure on \( A \), one needs only to construct a finite number of generating maps and verify the vanishing of a finite number of relation maps.
Below, we make these sets explicit for weights \( 1 \) and \( 2 \) and provide examples.
%; the case of weight \( 1 \) recovers the (strict) algebras and the weight 2 case is used in the next section to connect with the algebraic structures present in Yang-Mills theory.

\subsubsection{Weight 0}

The only generating map of degree \( 0 \) is \( m^0_1 \) and the only relation of weight \( 0 \) is
% differential
\[
\sM^0_{1} = \displaystyle m^0_1 \circ m^0_1 = \tfrac{1}{2}[m^0_1, m^0_1].
\]
Therefore, a \( \bv_{(0)} \)- or \( \bvbox_{(0)} \)-algebra is simply a differential graded vector space.

\subsubsection{Weight 1}

The set of generating maps of weight \( 1 \) is:
%of a \( \bv_{(1)} \)-algebra structure or a \( \bvbox_{(1)} \)-algebra structure on \( A \) is the same; it consists linear maps:
\[
m^1_1 \colon A \to A, \qquad
m^0_2 \colon A^{\ot 2} \to A, \qquad
m^0_{1,1} \colon A \ot A \to A,
\]
of degrees:
\[
\bars{m^1_1} = -1, \quad
\bars{m^0_2} = 0, \quad
\bars{m^0_{1,1}} = -1.
\]
They are required to satisfy the following symmetric properties:
\[
m^1_{2} \circ \big(\id - (12)\big) = 0, \quad
m^1_{1,1} \circ \big(\id - (12)\big) = 0.
\]
The relation maps of weight \( 0 \) are presented in \cref{f:weight0}.
Therefore, \( \bv_{(1)} \)- and \( \bvbox_{(1)} \)-algebras are respectively \( \bv \)- and \( \bvbox \)-algebras respectively.

%or, explicitly,
%\begin{align*}
%	m^0_2(a_1 \ot a_2) &= (-1)^{\bars{a_1} \bars{a_2}} \, m^0_2(a_2 \ot a_1), \\
%	m^0_{1,1}(a_1 \ot a_2) &= (-1)^{\bars{a_1} \bars{a_2}} \, m^0_{1,1}(a_2 \ot a_1).
%\end{align*}

%\subsubsection{Relation maps}
%
%To describe the relations satisfied by these algebras, we use the following compositions of generating maps, referred to as \emph{relation maps}, which are presented with a segmentation that will play a role in the next subsection.
\begin{figure}
	\renewcommand{\arraystretch}{1.25}
\begin{tabularx}{\textwidth}{|@{}r@{\hspace{2pt}}X@{}|}
	\hline
	% derivation of product
	\(\sM^0_{2}\) & \( = \displaystyle
	m^0_1 \circ_1 m^0_2
	- m^0_2 \circ_1 m^0_1
	- m^0_2 \circ_2 m^0_1
	= [m^0_1, m^0_2] \) \\

	\hline
	% commutativity of differential and codifferential
	\(\sM^1_1\) & \( = \displaystyle m^0_1 \circ m^1_1 + m^1_1 \circ m^0_1 \)
	\( = [m^0_1, m^1_1] \) \\

	\hline
	% derivation of the bracket
	\(\sM^0_{1,1}\) & \( = \displaystyle
	m^0_1 \circ m^0_{1,1}
	- m^0_{1,1} \circ_1 m^0_1
	- m^0_{1,1} \circ_2 m^0_1
	= [m^0_1, m^0_{1,1}] \) \\

	\hline
\end{tabularx}

	\caption{Relation maps of weight 2.}
	\label{f:weight0}
\end{figure}

\begin{figure}
	\renewcommand{\arraystretch}{1.25}
\begin{tabularx}{\textwidth}{|@{}r@{\hspace{2pt}}X@{}|}
	\hline
	% associativity of product
	\(\sM^0_{3}\) & \( = [m^0_1, m^0_3] - m^0_2 \circ_1 m^0_2 + m^0_2 \circ_2 m^0_2 \)
	\( = [m^0_1, m^0_3] - \frac{1}{2}[m^0_2, m^0_2] \) \anibal{Is this right?} \\

	\hline
	% leibniz
	\(\sM^0_{1,2}\) & \( = [m^0_1, m^0_{1,2}]
	- m^0_{2} \circ_1 m^0_{1,1}
	- (m^0_{2} \circ_2 m^0_{1,1}) \circ(12)
	+ m^0_{1,1} \circ_2 m^0_2 \) \\
	& \( = [m^0_1, m^0_{1,2}] - [m^0_2, m^0_{1,1}] \) \\

	\hline
	% jacobi
	\(\sM^0_{1,1,1}\) & \( =
	[m^0_1, m^0_{1,1,1}] - \big(m^0_{1,1} \circ_1 m^0_{1,1}\big) \circ \big(\id + (123) + (132)\big) \) \\

	\hline
	% bracket is a coderivation
	\(\sM^1_{1,1}\) & \( =
	[m^0_1, m^1_{1,1}]
	- m^1_1 \circ m^0_{1,1}
	+ m^0_{1,1} \circ_1 m^1_1
	+ m^0_{1,1} \circ_2 m^1_1 \) \\
	& \( = [m^0_1, m^1_{1,1}] - [m^1_1, m^0_{1,1}] \) \anibal{Why the fact that these are odd does not add a minus sign?}\\

	\hline
	% failure of product being a coderivation is the bracket
	\(\sM^1_2\) & \( =
	[m^0_1, m^1_2]
	+ m^1_1 \circ_1 m^0_2
	- m^0_2 \circ_1 m^1_1
	- m^0_2 \circ_2 m^1_1
	- m^0_{1,1} \) \\
	& \( = [m^0_1, m^1_2] + [m^1_1, m^0_2] - m^0_{1,1} \) \\

	\hline
	% codifferential
	\(\sM^2_1\) & \( =
	[m^0_1, m^2_1]
	+ m^1_1 \circ m^1_1 \) \\
%	\( = -\frac{1}{2} [m^1_1, m^1_1] \)\\
	\hline
\end{tabularx}
	\caption{Relation maps of weight 3.}
	\label{f:weight1}
\end{figure}

%\subsubsection{ \( \bv_{(1)} \)-algebras}
%
%A \defn{\( \bv_{(1)} \)-algebra} structure on \( A \) consists of the generating maps above, subject to the condition that all relation maps are identically zero.
%
%\subsubsection{ \( \bvbox_{(1)} \)-algebras}
%
%Similarly, a \defn{\( \bvbox_{(1)} \)-algebra} structure on \( A \) consists of the same data, subject only to the requirement that the relation maps \( \sM^0_1 \), \( \sM^0_2 \), and \( \sM^0_3 \) are identically zero.

\subsubsection{Level 2}

The subset of generating maps of weight \( 2 \) is:

\begin{align*}
	&m^0_{3}      \colon A^{\ot 3} \to A,
	&& m^0_{1,2}    \colon A \ot A^{\ot 2} \to A,
	&& m^0_{1,1,1}  \colon A \ot A \ot A \to A, \\
	&m^1_{1,1}    \colon A \ot A \to A,
	&& m^1_{2}      \colon A^{\ot 2} \to A,
	&& m^2_{1}      \colon A \to A,
\end{align*}
whose degrees are:
\begin{align*}
	&\bars{m^0_{3}} = -1,   && \bars{m^0_{1,2}} = -2,   && \bars{m^0_{1,1,1}} = -3, \\
	&\bars{m^1_{1,1}} = -3, && \bars{m^1_{2}}   = -2,   && \bars{m^2_{1}}     = -3.
\end{align*}
They are required to satisfy the following symmetric properties:
\begin{equation}\label{eq:symmetry_weight2}
	\begin{split}
		m^0_3 \circ \big(\id - (12) + (132)\big) &= 0, \qquad \Big(m^0_3 \circ \big(\id - (23) + (123)\big) = 0,\Big) \\
		m^0_{1,2} \circ \big(\id - (23)\big) &= 0, \qquad \Big(m^0_{2,1} - m^0_{1,2} \circ (123) = 0,\Big) \\
		m^0_{1,1,1} \circ \big(\id - (12)\big) &= 0, \qquad m^0_{1,1,1} \circ \big(\id - (23)\big) = 0, \\
		m^1_{1,1} \circ \big(\id - (12)\big) &= 0, \qquad m^1_{2} \circ \big(\id - (12)\big) = 0.
	\end{split}
\end{equation}

\noindent The relation maps of weight 2 are presented in \cref{f:weight2}.

\begin{figure}
	\renewcommand{\arraystretch}{1.25}
\begin{tabularx}{\textwidth}{|@{}r@{\hspace{2pt}}X@{}|}
	\hline
	\(\sM^0_4\) & \( = \displaystyle
	m^0_2 \circ_1 m^0_3 +m^0_3 \circ_2 m^0_2+m^0_2 \circ_2 m^0_3
	-m^0_3 \circ_3 m^0_2-m^0_3 \circ_1 m^0_2, \) \\

	\hline
	\(\sM^0_{2,2}\) & \( = \displaystyle
	- m^0_{1,2} \circ_1 m^0_2
	+ m^0_{2,1} \circ_3 m^0_2
	- (m^0_2 \circ_2 m^0_{1,2}) \circ (12)
	+ m^0_2 \circ_2 m^0_{1,2}
	- (m_2^0 \circ_2 m^0_{2,1}) \circ (123)
	+ m_2^0 \circ_1 m^0_{2,1}
	- m^0_3 \circ_2 m^0_{1,1}
	+ (m^0_3 \circ_3 m^0_{1,1}) \circ (123)
	- (m^0_3 \circ_3 m^0_{1,1}) \circ (23)
	- (m^0_3 \circ_2 m^0_{1,1}) \circ (1243)
	+ (m^0_3 \circ_1 m^0_{1,1}) \circ (23)
	- (m^0_3 \circ_1 m^0_{1,1}) \circ (234),
	\) \\

	\hline
	\(\sM^0_{1,3}\) & \( = \displaystyle
	(m_2^0 \circ_2 m^0_{1,2}) \circ (12)
	- m_2^0 \circ_1 m^0_{1,2}
	- m_{1,2} \circ_2 m_2^0
	+ m_{1,2} \circ_3 m_2^0 \) \\ & \( \quad
	- m_{1,1}^0 \circ_2 m^0_3
	- m_3^0 \circ_1 m^0_{1,1}
	- (m_3^0 \circ_2 m^0_{1,1}) \circ (12)
	- (m_3^0 \circ_3 m^0_{1,1}) \circ (321),
	\) \\

	\hline
	\(\sM^0_{1,1,2}\) & \( = \displaystyle
	m^0_{1,2}\circ_1 m^0_{1,1}
	+m^0_{1,2}\circ_2 m^0_{1,1}
	+\left(m^0_{1,2}\circ_3 m^0_{1,1}\right) \circ (23)
	+\left(m^0_{1,2}\circ_2 m^0_{1,1}\right) \circ (12)
	+\left(m^0_{1,2}\circ_3 m^0_{1,1}\right) \circ (321)
	- m^0_{1,1}\circ_2 m^0_{1,2}
	- \left(m^0_{1,1}\circ_2 m^0_{1,2}\right) \circ (12)
	+ m^0_{2}\circ_1 m^0_{1,1,1}
	+ \left(m^0_{2}\circ_1 m^0_{1,1,1}\right) \circ (34)
	- m^0_{1,1,1}\circ_3 m^0_{2}
	+ n^0_{1,1,2},
	\) \\

	\hline
	\(\sM^0_{1,1,1,1}\) & \( = \displaystyle
	- (m^0_{1,1} \circ_1 m^0_{1,1,1}) \circ \big(\id + (1234) + (13)(24) + (4321)\big) \) \\ & \( \quad
	- (m^0_{1,1,1} \circ_1 m^0_{1,1}) \circ \big(\id + (23) + (234) + (123) + (1342) + (13)(24)\big), \) \\

	\hline
	\(\sM^1_{3}\) & \( = \displaystyle
	- [m_1^1, m_3^0]
	- m_2^1 \circ_1 m_2^0 + m_2^1 \circ_2 m_2^0
	- m_2^0 \circ_1 m_2^1 + m_2^0 \circ_2 m_2^1
	+ m^0_{1,2} + m^0_{2,1}, \) \\

	\hline
	\(\sM^1_{1,2}\) & \( = \displaystyle
	- [m^1_1, m^0_{1,2}]
	+ m^0_{2} \circ_1 m^1_{1,1}
	+ (m^0_{2} \circ_2 m^1_{1,1}) \circ (12)
	- m^0_{1,1} \circ_2 m^1_2 \) \\ & \( \quad
	+ m^1_{2} \circ_1 m^0_{1,1}
	+ (m^1_{2} \circ_2 m^0_{1,1}) \circ (12)
	- m^0_{1,1} \circ_2 m^1_2
	+ m^0_{1,1,1},
	\) \\

	\hline
	\(\sM^1_{1,1,1}\) & \( = \displaystyle
	- [m^1_1, m^0_{1,1,1}]
	- (m^0_{1,1} \circ_1 m^1_{1,1}) \circ \big(\id + (123) + (321)\big) \) \\ & \( \quad
	- (m^1_{1,1} \circ_1 m^0_{1,1}) \circ \big(\id + (123) + (321)\big), \) \\

	\hline
	\(\sM^2_{1,1}\) & \( = \displaystyle
	- [m_1^1, m^1_{1,1}]
	- [m^2_1, m^0_{1,1}], \) \\

	\hline
	\(\sM^2_{2}\) & \( = \displaystyle
	- [m^1_1, m_2^1]
	- [m^2_1, m_2^0]
	+ m^1_{1,1}, \) \\

	\hline
	\(\sM^3_{1}\) & \( = \displaystyle
	- [m_1^1, m_1^2], \) \\

	\hline
\end{tabularx}
	\caption{Relation maps of weight 4 MISSING THE BOUNDARIES}
	\label{f:weight2}
\end{figure}