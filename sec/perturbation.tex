% !TEX root = ../ck1.tex

\section{Cyclic differential graded Lie algebras and classical field theory}

\subsection{Maurer--Cartan field theories}

....

\subsection{Scalar field theory}

We review the massless scalar \(\phi^3\) theory on flat Minkowski spacetime \(M\) as a concrete example of a Maurer--Cartan theory.
Let \(\cL\) be concentrated in degree \(0\) and \(1\) with \(\cL_0 \defeq \cO(M)\) and \(\cL_1 \coloneqq \mathrm{s}\cL_0\).
We define a differential and a degree \(1\) bracket by
\[
d(\phi) \coloneqq \Box \phi,
\qquad
[\phi_1,\phi_2] \coloneqq -g\,\phi_1\phi_2,
\]
where \(\Box\) is the wave operator and \(g\) is a (\textit{coupling}) constant.
The dg shifted Lie algebra \(\cL\) is naturally equipped with the pairing
\[
\langle \phi, E\rangle \coloneqq \int_M \phi(x)\,E(x)\,dx.
\]
With respect to this pairing, the Maurer--Cartan functional
\[
S[\phi] \coloneqq \tfrac12 \langle \phi, d\phi\rangle + \tfrac16 \langle \phi, [\phi,\phi]\rangle
\]
takes the explicit form
\[
S[\phi] = \int_M \Bigl( \tfrac12\,\phi(x)\,\Box\phi(x) - \tfrac{g}{3!}\,\phi(x)^3 \Bigr)\,dx,
\]
which is the classical action functional of the massless scalar \(\phi^3\) theory on \(M\).

\subsection{Classical elimination of an auxiliary field}

We consider a classical field theory with two \textit{fields} \(x\) and \(y\) and an \textit{action} of the form
\[
S(x,y) = S_0(x) + \tfrac12 \langle y, K y \rangle + S_{\mathrm{int}}(x,y).
\]
Here \(K\) is a linear operator, assumed invertible, and \(\langle \cdot, \cdot \rangle\) denotes a non-degenerate pairing.
We call \(G = K^{-1}\) the \textit{propagator} for \(y\).

We seek a description of the theory depending only on \(x\), obtained by eliminating \(y\) through its \textit{equation of motion}
\[
\frac{\delta S}{\delta y}(x,y) = K y + \frac{\delta S_{\mathrm{int}}}{\delta y}(x,y) = 0.
\]
This may be rewritten as the fixed-point equation
\[
y = -G \frac{\delta S_{\mathrm{int}}}{\delta y}(x,y).
\]
A formal solution \(y = y(x)\) can be constructed perturbatively by Picard iteration:
\[
y^{(0)} = 0, \qquad y^{(n+1)} = -G \frac{\delta S_{\mathrm{int}}}{\delta y}\bigl(x, y^{(n)}\bigr).
\]
The \textit{effective action} is defined by substitution:
\[
S_{\mathrm{eff}}(x) \coloneqq S\bigl(x, y(x)\bigr).
\]
By the chain rule, together with the fact that \(y(x)\) satisfies the equation of motion,
\[
\frac{\delta S_{\mathrm{eff}}}{\delta x}(x) = \left. \frac{\delta S}{\delta x}(x,y) \right|_{y = y(x)}.
\]
Thus the effective action reproduces the classical equations of motion for \(x\) once the auxiliary field \(y\) has been placed \textit{on shell}.

\subsection{Classical elimination in Maurer--Cartan theories}

We now consider a theory whose fields form a cyclic dg Lie algebra
\[
\big(\mathcal L, d, [-,-], \langle -,- \rangle \big),
\]
with action given by the Maurer--Cartan functional
\[
S_L(\phi) = \tfrac12 \langle \phi, d\phi \rangle + \tfrac16 \langle \phi, [\phi,\phi] \rangle.
\]
The Euler--Lagrange equation of this action is
\[
d\phi + \tfrac12 [\phi,\phi] = 0.
\]
Assume that \(\mathcal L\) is equipped with a contraction \((i,p,h)\) compatible with the pairing in the sense that the splitting
\[
L \cong i(\img  p) \oplus \ker p
\]
is orthogonal.
We will regard fields in \(i(\img  p)\) as retained degrees of freedom and fields in \(\ker p\) as auxiliary.
Writing a general field as
\[
\phi = i(x) + y, \qquad x \in \img  p,\ y \in \ker p,
\]
the Euler--Lagrange equation in the auxiliary directions is obtained by restricting variations to \(\ker p\), equivalently by projecting the Euler--Lagrange equation with
\[
q \coloneqq \mathrm{id}_L - ip.
\]
This yields
\[
q \Bigl( d(i(x)+y) + \tfrac12 [i(x)+y, i(x)+y] \Bigr) = 0
\]
or, equivalently,
\[
dy + q \Bigl( \tfrac12 [i(x)+y, i(x)+y] \Bigr) = 0.
\]
Imposing the \textit{gauge-fixing condition} \(y \in \ker (h)\) and applying \(h\) yields
\[
y = -h \, q \Bigl( \tfrac12 [i(x)+y, i(x)+y] \Bigr) = -h \Bigl( \tfrac12 [i(x)+y, i(x)+y] \Bigr)
\]
since \(\mathrm{id}_L = dh + hd + ip\) and \(hi = 0\).
This equation admits a formal solution \(y = y(x)\) obtained by Picard iteration:
\[
y^{(0)} = 0, \qquad
y^{(n+1)} = -h \big(\tfrac12 [i(x)+y^{(n)}, i(x)+y^{(n)}]\big).
\]
Explicitly, the formal solution \(y(x)\) is the sum over all planar binary trees, where each tree contributes the operation obtained by labeling leaves with \(i(x)\), internal vertices with \([-,-]\), and internal edges and root with \(h\).
Note that this formula treats \([-,-]\) as an ordered binary operation and does not invoke its antisymmetry.\anibal{Bring symmetries in and draw some examples.}
Explicitly,
\[
y(x) = \phi_2(i(x)) + \phi_3(i(x)) + \dotsb\,, \qquad \phi_n(i(x)) = ???
\]
Substituting \(y(x)\) into the original action defines the effective action on \(\img  p\),
\[
S_{\mathrm{eff}}(x) \coloneqq S_L\bigl( i(x) + y(x) \bigr).
\]
Using that \(\langle i(x), - \rangle = \langle x, p(-) \rangle\):
\begin{align*}
	S_{\mathrm{eff}}(x)
	&\,= \tfrac12 \langle x, l_1(x) \rangle + \tfrac16 \langle x, l_2(x) \rangle + \dotsb \\
	&\defeq \tfrac12 \langle x, p\,d\,i(x) \rangle + \tfrac16 \langle x, p[i(x), i(x)] \rangle + \dotsb
\end{align*}
We can polarize these expressions to get maps \(\ell_n \colon (\img p)^{\ot n} \to \ker p\),
\begin{align*}
	2! \, \ell_2(x_1, x_2) &= l_1(x_1 + x_2) - l_2(x_1) - l_2(x_2), \\
	3! \, \ell_3(x_1, x_2, x_3) &=  l_3(x_1 + x_2 + x_3) - l_3(x_1 + x_2) - l_3(x_1 + x_3) - l_3(x_2 + x_3) \\
	&\ + l_3(x_1) + l_3(x_2) + l_3(x_3), \\
	&\hspace*{6pt} \vdots
\end{align*}
which we recognize as the \(L_\infty\)-algebra structure on \(\ker p\) transfered via the contraction \((h, p, i)\).

\subsection{Scattering amplitudes}

%\subsubsection{Plane waves}
%
%To avoid analytic issues and to connect with scattering amplitudes, we restrict attention to
%finite linear combinations of \textit{plane waves}:
%\[
%\phi(x) = \sum_i \phi_i(k_i)\,e^{ik_i\cdot x},
%\qquad
%E(x) = \sum_i E_i(k_i)\,e^{ik_i\cdot x}.
%\]
%This corresponds to working in momentum space and restricting the Fourier transforms
%\(\hat\phi\) and \(\hat E\) to finite support.
%On such configurations the pairing evaluates to
%\[
%\langle \phi, E\rangle =
%(2\pi)^D \sum_{i,j}\delta^D(k_i+k_j)\,\phi_i(k_i)\,E_j(k_j).
%\]
%In this representation the kinetic operator acts diagonally,
%\[
%\Box e^{ik\cdot x} = -k^2\,e^{ik\cdot x},
%\]
%and the bracket is given by pointwise multiplication,
%\[
%[\phi_1,\phi_2](x) = -g\,\phi_1(x)\phi_2(x)
%= -g \sum_{i,j}\phi_{1,i}(k_i)\phi_{2,j}(q_j)\,e^{i(k_i+q_j)\cdot x}.
%\]
%In particular, the space of finite sums of plane waves is closed under the differential,
%the bracket, and the pairing, and therefore forms a cyclic dg Lie subalgebra.
%
%\anibal{check cyclicity}
%
%\subsubsection{On-shell projection}

%We consider the projections
%\[
%P_h\Big( \sum_i \psi_i\,e^{ik_i\cdot x}\Big) \coloneqq \sum_{k_i^2 = 0}\psi_i\,e^{ik_i\cdot x},
%\qquad
%(1-P_h)\Big(\psi(x)\Big) \coloneqq \sum_{k_i^2 \neq 0}\psi_i\,e^{ik_i\cdot x},
%\]
%for both fields and equations.
%
%
%the Green operator \(G\) is defined on plane waves with \(k^2 \neq 0\) by
%\[
%G\bigl(e^{ik\cdot x}\bigr) \coloneqq -\frac{1}{k^2}\,e^{ik\cdot x},
%\]
%and extends linearly to finite sums of such modes.
%The operator \(G\) is undefined on the kernel of \(\Box\), which consists of plane waves with \(k^2 = 0\) (on-shell modes).
%
%
%\[
%h \coloneqq G(1-P_h).
%\]
%It satisfies the contraction relations
%\[
%P_h^2 = P_h,
%\qquad
%P_h h = h P_h = 0,
%\qquad
%B_1 h + h B_1 = P_h - \mathrm{id}.
%\]
%
%\paragraph{Perturbative on-shell field.}
%We seek a formal solution \(\phi_\ast(\phi_0)\) with prescribed on-shell part \(\phi_0 \coloneqq P_h\phi_0\), \(\Box \phi_0 = 0\), of the form
%\[
%\phi_\ast(\phi_0) = \sum_{n\ge 0} g^n \phi_n(\phi_0) = \phi_0 + \phi_{\mathrm{n.l.}}(\phi_0),
%\qquad
%\phi_{\mathrm{n.l.}} \coloneqq \sum_{n\ge 1} g^n \phi_n.
%\]
%Using \(B_1 h + h B_1 = P_h - \mathrm{id}\), the field equation is equivalent to the fixed-point equation
%\[
%\phi_\ast
%=
%\phi_0 - \tfrac{g}{2}\,h(\phi_\ast^2),
%\]
%hence the recursion
%\[
%\phi_n = -\tfrac12 \sum_{k+\ell = n-1} h(\phi_k\phi_\ell),
%\qquad
%n\ge 1.
%\]
%The first few terms are
%\[
%\phi_1 = -\tfrac12 h(\phi_0^2),
%\qquad
%\phi_2 = \tfrac12 h\bigl(\phi_0\,h(\phi_0^2)\bigr),
%\]
%\[
%\phi_3
%=
%-\tfrac12 h\Bigl(\phi_0\,h\bigl(\phi_0\,h(\phi_0^2)\bigr)\Bigr)
%-\tfrac18 h\bigl(h(\phi_0^2)\,h(\phi_0^2)\bigr).
%\]
%Equivalently, \(\phi_{\mathrm{n.l.}}(\phi_0)\) is the sum over (planar) cubic tree diagrams with leaves labeled by \(\phi_0\), internal vertices labeled by multiplication, and internal edges labeled by \(h\), with the usual symmetry factors coming from expanding \((\phi_0+\phi_{\mathrm{n.l.}})^2\).
%
%\paragraph{Obstruction to solving the field equation.}
%Because \(h\) vanishes on on-shell momenta, \(\phi_\ast(\phi_0)\) does not solve the equation exactly in general.
%Indeed,
%\[
%\Box \phi_\ast - \tfrac{g}{2}\phi_\ast^2 = -\tfrac{g}{2}\,P_h(\phi_\ast^2),
%\qquad
%P_h\phi_\ast = \phi_0.
%\]
%
%\paragraph{On-shell action.}
%We define the on-shell functional by evaluation on the perturbative field,
%\[
%\bar S[\phi_0] \coloneqq S[\phi_\ast(\phi_0)].
%\]
%Using \(P_h\phi_0=\phi_0\) and \(P_h\phi_{\mathrm{n.l.}}=0\), one obtains the implicit form
%\[
%\bar S[\phi_0]
%=
%\tfrac12 \bigl\langle \phi_{\mathrm{n.l.}}(\phi_0),\,\phi_{\mathrm{n.l.}}(\phi_0)\bigr\rangle
%-
%\tfrac{g}{3!}\Bigl\langle \phi_0 + \phi_{\mathrm{n.l.}}(\phi_0),\,\bigl(\phi_0+\phi_{\mathrm{n.l.}}(\phi_0)\bigr)^2\Bigr\rangle.
%\]
%
%\paragraph{Transferred brackets on on-shell fields.}
%The functional \(\bar S[\phi_0]\) determines higher brackets \(\bar B_n\) by polarization of its variation,
%\[
%\frac{\delta \bar S}{\delta \phi_0}
%\coloneqq
%\tfrac12 \bar B_2(\phi_0,\phi_0)
%+
%\tfrac{1}{3!}\bar B_3(\phi_0,\phi_0,\phi_0)
%+
%\dotsb
%+
%\tfrac{1}{n!}\bar B_n(\phi_0,\dotsc,\phi_0)
%+
%\dotsb\, .
%\]
%The first ones read
%\[
%\bar B_2(\phi_0,\phi_0) = -g\,P_h(\phi_0^2),
%\qquad
%\bar B_3(\phi_0,\phi_0,\phi_0) = 3g^2\,P_h\bigl(\phi_0\,h(\phi_0^2)\bigr),
%\]
%\[
%\bar B_4(\phi_0,\phi_0,\phi_0,\phi_0)
%=
%-g^3\,P_h\Bigl(\tfrac12 \phi_0\,h\bigl(\phi_0\,h(\phi_0^2)\bigr) + 3\,h(\phi_0^2)\,h(\phi_0^2)\Bigr).
%\]
%Diagrammatically, the brackets \(\bar B_n\) are given by the sum of all inequivalent cubic trees with \(n\) labeled external legs and one output, each counted once.
%Equivalently, they are obtained by homotopy transfer along the contraction with projection \(p = P_h\), inclusion \(\iota\) the tautological inclusion of on-shell plane waves, and homotopy \(h = G(1-P_h)\).



