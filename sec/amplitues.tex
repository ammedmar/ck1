
\section{Amplitudes}

We work in Minkowski space \(\R^{1+d}\) with metric \(\eta_{\mu\nu} = \mathrm{diag}(-,+,\dots,+)\).
The d'Alembertian is \(\Box \defeq \eta^{\mu\nu}\partial_\mu \partial_\nu\).

\subsection{Plane waves}

Any tempered distribution admits a Fourier representation
\[
f(x) = \int_{\R^{1,d}} \hat f(p)\, e^{ip\cdot x}\,\mathrm{d}p,
\qquad p\cdot x = \eta_{\mu\nu} p^\mu x^\nu,
\]
up to normalization constants.
For scattering purposes we restrict to the subspace of finite sums:
\[
W \defeq \Big\{ \phi(x) = \sum_j \phi_j\, e^{i p_j\cdot x}\Big\}.
\]

A \defn{plane wave} \(e^{i p\cdot x}\) has \defn{momentum} \(p\) and satisfies
\[
\Box(e^{i p\cdot x}) = -p^2 e^{i p\cdot x}, \qquad p^2 \defeq p\cdot p.
\]
It is called \defn{on-shell} if \(p^2=0\), and \defn{off-shell} otherwise.
Let
\[
W_0 \defeq \ker \Box \cap W
\]
be the span of on-shell plane waves,
and let \(\rP_0\colon W\to W_0\) denote the projection onto them.

The \defn{propagator} \(G\) is defined as the inverse of \(\Box\) on the off-shell subspace:
\[
\big(G\circ (\id-\rP_0)\big)(\phi)
= \sum_{p_j^2\neq 0} \frac{1}{p_j^2} \phi_j\, e^{i p_j\cdot x}.
\]




\newpage
\subsection{old}
Let us consider a \( \bvbox \) algebra \( (A, d, d^\star, \mu) \) and a cyclic Lie algebra \( (\fg, [-,-], \angles{-,-}) \).
We are interested in transfering the dg Lie algebra \( A \ot \fg \) to \( \ker \square \ot \id_{\fg} \).
If \( d \)

We want to show that the scattering amplitudes associated with the Lie algebra \( \fg \ot A \) can be computed using the

%Let us consider a differential graded commutative and associative algebra \( (A, d, \mu) \).
%A choice of codifferential \( d^\star \) defines a dg shifted lie algebra

%Let us consider a \( \bvbox \)-algebra \( (A, d, d^\star, \mu) \), i.e. a bidifferential complex \( (A, d, d^\star) \) together with a commutative product



Little summary of the tree-level amplitude via HHT paper.

Our goal is to transfer the Yang-Mills $L_\infty$ structure from $\mathcal{X}_{\mathrm{YM}}$ to a space $\bar{\mathcal{X}}_{\mathrm{YM}}$, where the fields $\bar{A}_\mu \in \bar{\mathcal{X}}_0$ are on-shell and transverse:
\begin{equation}
	\Box \bar{A}_\mu = 0\,, \qquad \partial \cdot \bar{A} = 0\,,
	\tag{3.20}
\end{equation}
which implies $\bar{B}_1(\bar{\mathcal{A}}) = 0$ for the differential on $\bar{\mathcal{X}}_{\mathrm{YM}}$. The fields $\bar{A}_\mu$ are still subject to residual on-shell gauge transformations $\delta \bar{A}_\mu = \partial_\mu \bar{\Lambda}$ for parameters obeying $\Box \bar{\Lambda} = 0$, meaning that the space $\bar{\mathcal{X}}_{-1}$ consists of harmonic gauge parameters. Furthermore, the on-shell and transverse conditions on $\bar{A}_\mu$ suggest that the space of currents $\bar{\mathcal{X}}_1$ contains on-shell vectors defined modulo gradients, while the space of Noether identities $\bar{\mathcal{X}}_2$ consists of harmonic scalars:
\begin{equation}
	\Box \bar{\mathcal{J}}_\mu = 0\,, \qquad \bar{\mathcal{J}}_\mu \sim \bar{\mathcal{J}}_\mu + \partial_\mu \Phi\,, \qquad \Box \bar{\mathcal{N}} = 0\,.
	\tag{3.21}
\end{equation}
The equivalence class condition defining $\bar{\mathcal{X}}_1$ ensures that the pairing between an on-shell and transverse field $\bar{A}_\mu$ with a current $\bar{\mathcal{J}}_\mu$ remains non-degenerate. Indeed, if $\int dx \, \mathrm{Tr}(\bar{A}^\mu \bar{\mathcal{J}}_\mu) = 0$
