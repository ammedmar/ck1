% !TEX root = ../ck_dual.tex

\section{Bi-differential algebras}

In all the geometric examples of bi-differential complexes given in \cref{ss:geometric examples}, there is also a product structure, the wedge product of differential forms.

In this section we study the relationship between this product and the bi-differential structure.

We will start with a perspective that assumes no compatibility between the bi-differential structure and a degree \( 0 \) (graded) commutative and associative product.

\subsection{Shifted Lie algebras}

A \defn{shifted Lie algebra} is a differential complex \( (V, d) \) together with a skew-symmetric bilinear bracket
\[
[-,-] \colon V \otimes V \to V
\]
of degree \( 1 \) satisfying the graded Jacobi identity and compatibility with the differential in the form
\[
d [x,y] = [dx, y] + (-1)^{|x|+1} [x, dy]
\]
for all homogeneous \( x, y \in V \).

\subsection{Shifted Lie infinity algebras}

A \defn{shifted \( L_\infty \)-algebra} (also called an \( s\mathcal{L}ie_\infty \)-algebra) is a graded vector space \( V \) equipped with a collection of graded symmetric multilinear maps
\[
\ell_n \colon V^{\otimes n} \to V \quad \text{of degree } 1 \text{ for all } n \geq 1,
\]
satisfying the higher Jacobi identities:
\[
\sum_{i+j = n+1} \sum_{\sigma \in \mathrm{Sh}(j,n-j)} \epsilon(\sigma; v_1, \dots, v_n) \, \ell_i\big(\ell_j(v_{\sigma(1)}, \dots, v_{\sigma(j)}), v_{\sigma(j+1)}, \dots, v_{\sigma(n)}\big) = 0,
\]
for all \( n \geq 1 \) and homogeneous \( v_1, \dots, v_n \in V \).

Here, \( \mathrm{Sh}(j,n-j) \) denotes the set of \((j,n-j)\)-shuffles and \( \epsilon(\sigma; -) \) is the Koszul sign rule associated to the permutation \( \sigma \).

\subsection{Moduli of \( s\mathcal{L}ie_\infty \)}

Given a shifted \( L_\infty \)-algebra \( V \), a \defn{Maurer–Cartan element} is a degree \( 0 \) element \( x \in V^0 \) satisfying the equation
\[
\sum_{n \geq 1} \frac{1}{n!} \ell_n(x, \dots, x) = 0,
\]
where the sum is finite if \( V \) is nilpotent or suitably filtered.

The \defn{moduli space of Maurer–Cartan elements}, denoted \( \mathrm{MC}(V)/\sim \), is the set of Maurer–Cartan elements modulo the gauge equivalence relation induced by the \( L_\infty \)-algebra structure. This moduli space encodes infinitesimal deformations governed by \( V \).

When \( V \) arises from a deformation problem, this moduli space is a formal moduli space in the sense of derived deformation theory.

Gauge elements...

\subsection{}

Product as a gauge element...

\subsection{Koszul hierarchy}

...








