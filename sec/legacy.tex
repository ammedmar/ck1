\begin{comment}
	Revisiting these conditions using the box operator, we observe that a bi-differential complex \( (V, d, d^\star) \) is a bi-contraction if and only if \( \square = \id \) on \( (\img d + \img d^\star) \).\footnote{
		Explicitly, \( d = d \circ d^\star \circ d \) if and only if \( d = \square \circ d \), which holds if and only if \( \square|_{\img d} = \id \).
		A similar argument applies to the condition \( d^\star = d^\star \circ d \circ d^\star \), which is equivalent to \( \square|_{\img d^\star} = \id \).
	}

	Since \( \img\square \subseteq \img d + \img d^\star \) the bi-contraction property implies \( \img\square =  \img d + \img d^\star\).
	As we will see, this weaker condition suffices to construct a \textit{Hodge-type decomposition} as the one constructed in \cref{thm:bi-contraction}.

	\begin{definition}
		A \defn{\( dd^\star\! \)-complex} as a bi-differential complex \( (V, d, d^\star) \) satisfying the \defn{\( dd^\star\! \)-condition}:
		\[
		\img \square = \img d + \img d^\star.
		\]
	\end{definition}

	\begin{lemma}\label{thm:d dstar lemma}
		Let \( (V, d, d^\star) \) be a Green bi-differential complex which is either bounded below or above.
		Then, there exists a bi-differential morphism \( G \colon \img\square \to V \) with \( \square \circ G = \id_{\img\square} \) such that \( (V, d, G \circ d^\star) \) and \( (V, G \circ d, d^\star) \) are bi-contractions.
	\end{lemma}
\end{comment}


Let \( \pi = \tfrac{1}{3}(\id + (123) + (132)) \in \R[\sym_3] \) and denote, for any map \( \tau \colon \cK^{\ot 3} \to \cK\),
\[
\tau_{\mathrm{s}} = \tau \circ \pi
\quad\text{and}\quad
\tau_{\mathrm{h}} = \tau - \tau_\mathrm{s}.
\]

\noindent The map \( m_{3\mathrm{h}} \) is defined by
\[
\theta_\mu \ot \theta_\nu \ot \theta_\rho
\mapsto
c\theta_\mu \eta_{\nu\rho} - c\theta_\nu \eta_{\mu\rho}.
\]

\noindent The authors also explicitly define maps \( b_3, \theta_3 \colon \cK^{\ot 3} \to \cK\) and claim they collectively satisfy:
\begin{align*}
	&m_1^2 = 0, \quad b^2 = 0, \quad [m_1, b] = \square, \\
	&[m_1, m_2] = 0, \quad m_2 m_2 (1 - \pi) = [m_1, m_{3\mathrm{h}}], \\
	&b_2 = [b, m_2], \quad [m_1, b_2] = [\square, m_2], \\
	&b_2 m_2 + m_2 b_2 (1 - 3\pi) = [m_1, \theta_3] + m_{3\mathrm{h}} (d_\square - 3 d_s \pi), \\
	&3 b_2 b_2 \pi + [m_1, b_3] + 3 \theta_3 d_s \pi = 0, \\
	&\theta_{3\mathrm{h}} + [b, m_{3\mathrm{h}}] = 0, \quad b_3 + [b, \theta_{3\mathrm{s}}] = 0.
\end{align*}
where \[
d_\square =
\overbrace{2(\partial^\mu \ot \partial_\mu \ot \id)}^{d_s}
+ 2(\id \ot \partial^\mu \ot \partial_\mu)
+ 2(\partial^\mu \ot \id \ot \partial_\mu).
\]

\hfil

\subsection{Weight 1}

The set of generating maps of weight equal to \( 1 \) is:
%of a \( \bv_{(1)} \)-algebra structure or a \( \bvbox_{(1)} \)-algebra structure on \( A \) is the same; it consists linear maps:
\begin{align*}
	&m^0_1 \colon A \to A, && m^1_1 \colon A \to A, \\
	&m^0_2 \colon A^{\ot 2} \to A, && m^0_{1,1} \colon A \ot A \to A,
\end{align*}
of degrees:
\[
\bars{m^0_1} = 1, \quad
\bars{m^1_1} = -1, \quad
\bars{m^0_2} = 0, \quad
\bars{m^0_{1,1}} = -1.
\]
They are required to satisfy the following symmetric properties:
\[
m^1_{2} \circ \big(\id - (12)\big) = 0, \quad
m^1_{1,1} \circ \big(\id - (12)\big) = 0.
\]
The relation maps of weight less than or equal to \( 0 \) are presented in \cref{f:weight0}.


%The operads \( \bv \) and \( \bvbox_\infty \) are equipped with natural filtrations and surjective maps
%\[
\begin{tikzcd}[column sep=small]
	\bvbox_{(1)} \arrow[r, hook] \arrow[d] &
	\bvbox_{(2)} \arrow[r, hook] \arrow[d] &
	\bvbox_{(3)} \arrow[r, hook] \arrow[d] &
	\dotsb \arrow[r, hook] &
	\bvbox_\infty \arrow[d] \\
	%
	\bv_{(1)} \arrow[r, hook] &
	\bv_{(2)} \arrow[r, hook] &
	\bv_{(3)} \arrow[r, hook] &
	\dotsb \arrow[r, hook] &
	\bv_\infty
\end{tikzcd}

...




\subsection{From Bruno}



\begin{theorem}
	A \emph{coexact homotopy BV-algebra} is dg module $(A, d_A)$ equipped
	with two families of maps
	\begin{align*}
		&m_{p_1,\dots,p_k}^t \colon
		{A^{\ot p_1}}\odot\dots\odot {A^{\ot p_k}}
		\longrightarrow A~,\qquad t\geqslant 0~,\   k\geqslant 1~, \ \text{and} \ \ p_1,\dots , p_k\geqslant 1~,\\
		&n_{p_1,\dots,p_k}^t \colon
		{A^{\ot p_1}}\odot\dots\odot {A^{\ot p_k}}
		\longrightarrow A~,\qquad
		t\geqslant 0~,\   k\geqslant 1~, \ p_1,\dots , p_k\geqslant 1~,
		\ \text{and} \ \ t+k\geqslant 2~,
	\end{align*}
	of respective degrees degrees
	\[\left| m_{p_1,\dots,p_k}^t \right| =p_1+\cdots+p_k+k+2t-3 \qquad \text{and} \qquad
	\left| n_{p_1,\dots,p_k}^t \right| =p_1+\cdots+p_k+k+2t-4~,\]
	with $m_1^0=d_A$~,
	and such that, for any $1\leqslant i \leqslant k$ and any $1\leqslant p \leqslant p_i-1$, the following signed sum of non-trivial shuffle actions vanish:
	\begin{align*}
		&\sum_{\sigma \in \mathrm{Sh}(p,p_i-p) }\mathrm{sgn}(\sigma)\,
		m_{p_1,\dots,p_k}^t  (\id_{p_1} \odot \cdots \odot \sigma \odot \cdots \odot \id_{p_k})=0~, \\
		&\sum_{\sigma \in \mathrm{Sh}(p,p_i-p) }\mathrm{sgn}(\sigma)\,
		n_{p_1,\dots,p_k}^t  (\id_{p_1} \odot \cdots \odot \sigma \odot \cdots \odot \id_{p_k})=0~.
	\end{align*}
	They satisfy the following two types of relations
	\begin{align*}\tag{$\mathsf{R}^t_{p_1,\dots,p_k}$}
		&\sum_{\substack{
				0\leqslant s \leqslant t
				\\[0.5mm]
				I\sqcup J=\{1,\dots,k\}
				\\[0.5mm]
				I=\{i_1, \dots, i_a\}\neq\emptyset \\[0.5mm]
				J=\{j_1, \dots, j_b\}
		}}
		\sum_{
			\substack{
				q_1,\dots,{q}_a\geqslant 1 \\[0.5mm]
				(q_1, \dots, q_a)\leqslant
				(p_1, \dots, p_{i_a})
		}}
		(-1)^{\varepsilon+\varepsilon'+\varepsilon''}
		\,m^{s}_{p',p_{j_1},\dots,p_{j_b}}
		\left(
		m^{t-s}_{\frac{p_1, \dots, p_{i_a}}{q_1, \dots, q_a}}
		(w_{i_1} \odot \dots \odot w_{i_a})\odot w_{j_1} \odot \dots \odot w_{j_b}
		\right)\\
		&\quad
		-\sum_{i=1}^k
		\sum_{1\leqslant p\leqslant p_i-1}(-1)^{\varepsilon'''}
		\,m^{t-1}_{p_1,\dots, p,p_i-p,\dots,p_k}
		\left(w_1\odot\dots\odot a_{[P_i+1, P_{i}+p]}
		\odot a_{[P_{i}+p+1,P_{i+1}]}\odot\dots\odot w_k\right)
		\\
		&\quad-n^t_{p_1, \dots, p_k}(w_1\odot \cdots \odot w_k)=0~,
	\end{align*}
	where the last term is present only for $t+k\geqslant 2$ and
	under the same notations as in \cref{thm:BVinfini}, and %%%%%%%%%%%%%%%%%%%%%%%%
	\begin{align*}\tag{$\mathsf{S}^t_{p_1,\dots,p_k}$}
		&\sum_{\substack{
				0\leqslant s \leqslant t
				\\[0.5mm]
				I\sqcup J=\{1,\dots,k\}
				\\[0.5mm]
				I=\{i_1, \dots, i_a\}\neq \emptyset \\[0.5mm]
				J=\{j_1, \dots, j_b\}
		}}
		\sum_{
			\substack{
				q_1,\dots,{q}_a\geqslant 1 \\[0.5mm]
				(q_1, \dots, q_a)\leqslant
				(p_1, \dots, p_{i_a})
		}}
		(-1)^{\varepsilon+\widetilde{\varepsilon}'+\varepsilon''}
		\,m^{s}_{p',p_{j_1},\dots,p_{j_b}}
		\left(
		n^{t-s}_{\frac{p_1, \dots, p_{i_a}}{q_1, \dots, q_a}}
		(w_{i_1} \odot \dots \odot w_{i_a})\odot w_{j_1} \odot \dots \odot w_{j_b}
		\right)\\
		&-\sum_{\substack{
				0\leqslant s \leqslant t
				\\[0.5mm]
				I\sqcup J=\{1,\dots,k\}
				\\[0.5mm]
				I=\{i_1, \dots, i_a\}\neq \emptyset \\[0.5mm]
				J=\{j_1, \dots, j_b\}
		}}
		\sum_{
			\substack{
				q_1,\dots,{q}_a\geqslant 1 \\[0.5mm]
				(q_1, \dots, q_a)\leqslant
				(p_1, \dots, p_{i_a})
		}}
		(-1)^{\varepsilon+\varepsilon'+\varepsilon''}
		\,n^{s}_{p',p_{j_1},\dots,p_{j_b}}
		\left(
		m^{t-s}_{\frac{p_1, \dots, p_{i_a}}{q_1, \dots, q_a}}
		(w_{i_1} \odot \dots \odot w_{i_a})\odot w_{j_1} \odot \dots \odot w_{j_b}
		\right)\\
		&\quad
		-\sum_{i=1}^k
		\sum_{1\leqslant p\leqslant p_i-1}(-1)^{\varepsilon'''}
		\,n^{t-1}_{p_1,\dots, p,p_i-p,\dots,p_k}
		\left(w_1\odot\dots\odot a_{[P_i+1, P_{i}+p]}
		\odot a_{[P_{i}+p+1,P_{i+1}]}\odot\dots\odot w_k\right)=0~,
	\end{align*}
	for $t+k\geqslant 2$, where
	$\widetilde{\varepsilon}'=(q_1+\cdots+q_a-a)(p_{j_1}+\cdots+p_{j_b}-b)$~.
\end{theorem}


\begin{proof}
	\anibal{TBC}
\end{proof}


The first symmetries are :


\[m_2(a_1\ot a_2)=+ m_2(a_2\ot a_1)\]
\[m_3(a_1\ot a_2 \ot a_3)- m_3(a_2\ot a_1 \ot a_3)  + m_3(a_2\ot a_3 \ot a_1) =0\]
1-2 shuffles
\[m_3(a_1\ot a_2 \ot a_3)- m_3(a_1\ot a_3 \ot a_2)  + m_3(a_1\ot a_3 \ot a_2) =0\]
2-1 shuffles


\[m_{1,1}(a_1\ot a_2)=+ m_{1,1}(a_2\ot a_1)\]
\[m_{1,1,1}(a_1\ot a_2\ot a_3)=+ m_{1,1,1}(a_2\ot a_1\ot a_3)=+ m_{1,1,1}(a_2\ot a_1\ot a_3)\]
In plain words, the symmetry is .... commutativity between blocks and "sum shuffle vanishing" inside blocks.
Also
\[m^t_{p_{\sigma(1)}, \dots, p_{\sigma(k)}}=\left(m^t_{p_1, \dots, p_k}\right)^{\widetilde{\sigma}}.\]


\medskip


Let us make the first relations explicit.


\begin{description}
	\item[$\left(\mathsf{R}^0_{1}\right)$] $m^0_1 \circ_1 m_1^0=\left(d_A\right)^2=0$, saying that $m_1^0=d_A$ is a differential.\\


	\item[$\left(\mathsf{R}^0_{2}\right)$] $\partial \left(m^0_2\right)= (\left[m^0_1, m_2^0\right])=0$, saying that $d_A$ is a derivation with respect to the product $m^0_2$.


	\item[$\left(\mathsf{R}^0_{3}\right)$] $\partial \left(m^0_3\right)=m^0_2\circ_1 m^0_2 - m^0_2 \circ_2 m^0_2$, saying that the product
	$m^0_2$ is associative up to homotopy $m_3^0$.


	\item[$\left(\mathsf{R}^0_{4}\right)$] $\partial \left(m^0_4\right)=
	m^0_2\circ_1 m^0_3 +m^0_3\circ_2 m^0_2+m^0_2\circ_2 m^0_3
	-m^0_3\circ_3 m^0_2-m^0_3\circ_1 m^0_2$, saying that the two operations $m^0_2$ and $m_3^0$ satisfy a relation up to the homotopy $m^0_4$~.


	\item[$\left(\mathsf{R}^0_{1,1}\right)$] $\partial \left(m^0_{1,1}\right)=n^0_{1,1}$,
	saying that the operation $n^0_{1,1}$ measures the obstruction for $d_A$ to be a derivation with respect to the shifted bracket $m^0_{1,1}$.


	\item[$\left(\mathsf{R}^0_{1,1,1}\right)$] $\partial \left(m^0_{1,1,1}\right)=
	-m^0_{1,1}\circ_1 m^0_{1,1}-\left(m^0_{1,1}\circ_1 m^0_{1,1}\right)^{(123)}-\left(m^0_{1,1}\circ_1 m^0_{1,1}\right)^{(321)}
	+n^0_{1,1,1}$,
	saying that the operation $n^0_{1,1,1}$ measures the obstruction for $m^0_{1,1,1}$ to be a homotopy for the (shifted) Jacobi relation for
	the shifted bracket $m^0_{1,1}$.


	\item[$\left(\mathsf{R}^0_{1,1,1,1}\right)$] $\partial \left(m^0_{1,1,1}\right)=
	-\left(m^0_{1,1}\circ_1 m^0_{1,1,1}\right)^{\id+(1234)+(13)(24)+(4321)}
	-\left(m^0_{1,1,1}\circ_1 m^0_{1,1}\right)^{\id+(23)+(234)+(123)+(1342)+(13)(24)}
	+n^0_{1,1,1,1}$,
	saying that the operation $n^0_{1,1,1,1}$ measures the obstruction for $m^0_{1,1,1,1}$ to be a homotopy for the shifted $L_\infty$-algebra structure provided by the operation $m^0_{1,1}$ and $m^0_{1,1,1}$.


	\item[$\left(\mathsf{R}^0_{1,2}\right)$] $\partial \left(m^0_{1,2}\right)=
	m^0_{2}\circ_1 m^0_{1,1}
	+\left(m^0_{2}\circ_2 m^0_{1,1}\right)^{(12)}
	-m^0_{1,1}\circ_2 m^0_2
	+n^0_{1,2}$,
	saying that the operation $n^0_{1,2}$ measures the obstruction for the product $m^0_{2}$ and the shifted bracket  $m^0_{1,1}$  to satisfy the Leibniz relation up to the homotopy $m^0_{1,2}$.


	\item[$\left(\mathsf{R}^1_{1}\right)$] $\partial \left(m^1_{1}\right)=n^1_{1}$,
	saying that the operator $n^1_{1}$ measures the obstruction for $d_A$ to be commute with the operator $m^1_{1}$.


	\item[$\left(\mathsf{R}^2_{1}\right)$] $\partial \left(m^2_{1}\right)=-\left(m^1_1\right)^2+n^2_{1}$,
	saying that the operator $n^2_{1}$ measures the obstruction for the operator $m^2_1$ to be a homotopy for the square-zero relation of
	$m^1_1$.


	\item[$\left(\mathsf{R}^1_{1,1}\right)$]
	$\partial \left(m^1_{1,1}\right)=
	-\left[m^1_1, m^0_{1,1} \right]
	+n^1_{1,1}=
	-m_1^1 \circ_1 m^0_{1,1}
	-m^0_{1,1}\circ_1 m_1^1
	-m^0_{1,1}\circ_2 m_1^1
	+n^1_{1,1}$,
	saying that the operator $n^1_{1,1}$ measures the obstruction for the operation $m^1_{1,1}$ to be a homotopy for the
	commutation relation between the operator
	$m_1^1$ and the shifted bracket $m^0_{1,1}$.


	\item[$\left(\mathsf{R}^1_{2}\right)$] $\partial \left(m^1_{2}\right)=
	-\left[m^1_1, m^0_{2} \right] + m^0_{1,1}
	+n^1_2=
	-m_1^1 \circ_1 m^0_2
	+m^0_2\circ_1 m_1^1
	+m^0_2\circ_2 m_1^1
	+ m^0_{1,1}
	+n^1_2$,
	saying that the operator $n^1_{2}$ measures the obstruction for the operation $m^1_2$ to be a homotopy for the
	relation giving  the shifted bracket $m^0_{1,1}$ as the commutator between the operator
	$m_1^1$ and the product $m^0_2$.\\


	\item[$\left(\mathsf{R}^1_{3}\right)$] $\partial \left(m^1_{3}\right)=-\left[m_1^1, m_3^0\right]
	-m_2^1\circ_1 m_2^0 +m_2^1\circ_2 m_2^0
	-m_2^0\circ_1 m_2^1 +m_2^0\circ_2 m_2^1
	+m^0_{1,2}-+m^0_{2,1}+n^1_3~.$


	\item[$\left(\mathsf{R}^3_{1}\right)$] $\partial \left(m^3_{1}\right)=-\left[m_1^1, m_1^2\right]+n^3_1$~.


	\item[$\left(\mathsf{R}^2_{1,1}\right)$] $\partial \left(m^2_{1,1}\right)=-\left[m_1^1, m^1_{1,1}\right]
	-\left[m^2_1, m^0_{1,1}\right]+n^2_{1,1}$~.


	\item[$\left(\mathsf{R}^2_{2}\right)$] $\partial \left(m^2_{2}\right)=
	-\left[m^1_1, m_2^1\right]
	-\left[m^2_1, m_2^0\right]
	+m^1_{1,1}+n^2_2
	$~.


	\item[$\left(\mathsf{R}^1_{1,1,1}\right)$] $\partial \left(m^1_{1,1,1}\right)=
	-\left[m^1_1, m^0_{1,1,1}\right]
	-\left(m^0_{1,1}\circ_1 m^1_{1,1}\right)^{\id+(123)+(321)}
	-\left(m^1_{1,1}\circ_1 m^0_{1,1}\right)^{\id+(123)+(321)}
	+n^1_{1,1,1}$~.


	\item[$\left(\mathsf{R}^1_{1,2}\right)$] $\partial \left(m^1_{1,2}\right)=
	-\left[m^1_1, m^0_{1,2}\right]
	+m^0_{2}\circ_1 m^1_{1,1}
	+\left(m^0_{2}\circ_2 m^1_{1,1}\right)^{(12)}
	-m^0_{1,1}\circ_2 m^1_2
	+m^1_{2}\circ_1 m^0_{1,1}
	+\left(m^1_{2}\circ_2 m^0_{1,1}\right)^{(12)}
	-m^0_{1,1}\circ_2 m^1_2
	+m^0_{1,1,1}
	+n^1_{1,2}$~.


	\item[$\left(\mathsf{R}^1_{1,3}\right)$] $\partial \left(m^1_{1,3}\right)=
	\left(m_2^0\circ_2 m^0_{1,2}\right)^{(12)}
	-m_2^0\circ_1 m^0_{1,2}
	-m_{1,2}\circ_2 m_2^0
	+m_{1,2}\circ_3 m_2^0
	-m_{1,1}^0\circ_2 m^0_3
	-m_3^0\circ_1 m^0_{1,1}
	-\left(m_3^0\circ_2 m^0_{1,1}\right)^{(12)}
	-\left(m_3^0\circ_3 m^0_{1,1}\right)^{(321)}
	+n^0_{1,3}
	$~.


	\item[$\left(\mathsf{R}^1_{2,2}\right)$] $\partial \left(m^1_{2,2}\right)=
	-m_{1,2}\circ_1 m^0_2
	+m_{2,1}\circ_3 m^0_2
	-\left(m^0_2 \circ_2 m^0_{1,2}\right)^{(12)}
	+m^0_2 \circ_2 m^0_{1,2}
	-\left(m_2^0 \circ_2 m^0_{2,1}\right)^{(123)}
	+m_2^0 \circ_1 m^0_{2,1}
	-m^0_3\circ_2 m^0_{1,1}
	+\left(m^0_3\circ_3 m^0_{1,1}\right)^{(123)}
	-\left(m^0_3\circ_3 m^0_{1,1}\right)^{(23)}
	-\left(m^0_3\circ_2 m^0_{1,1}\right)^{(1243)}
	+\left(m^0_3\circ_1 m^0_{1,1}\right)^{(23)}
	-\left(m^0_3\circ_1 m^0_{1,1}\right)^{(234)}
	+n^0_{2,2}
	$~.
\end{description}


\bigskip


\begin{description}
	\item[$\left(\mathsf{S}^1_{1}\right)$] $\partial\left(n^1_{1}\right)=0$, saying that $d_A$ is a derivation with respect to the obstruction
	$n^0_{1,1}$.


	\item[$\left(\mathsf{S}^0_{1,1}\right)$] $\partial\left(n^1_{1,1}\right)=0$, saying that $d_A$ is a derivation with respect to the obstruction
	$n^0_{1,1}$.


	\item[$\left(\mathsf{S}^1_{1,1}\right)$]
	$\partial\left(n^1_{1,1}\right)=\left[n^1_1, m^0_{1,1}\right]-\left[m^1_1, n^0_{1,1}\right]=
	n_1^1 \circ_1 m^0_{1,1}
	-m^0_{1,1}\circ_1 n_1^1
	-m^0_{1,1}\circ_2 n_1^1
	-m_1^1 \circ_1 n^0_{1,1}
	+n^0_{1,1}\circ_1 m_1^1
	+n^0_{1,1}\circ_2 m_1^1
	$,
	saying that $n^1_{1,1}$ is a homotopy for the equality between
	the commutator of $n^1_1$ with $m^0_{1,1}$
	and the commutator of $m^1_1$ with $n^0_{1,1}$.


	\item[$\left(\mathsf{S}^1_{2}\right)$]
	$\partial\left(n^1_{2}\right)=\left[n^1_1, m^0_2\right]+\textcolor{red}{must\ be\ -}n^0_{1,1}
	=
	n_1^1 \circ_1 m^0_{2}
	-m^0_{2}\circ_1 n_1^1
	-m^0_{2}\circ_2 n_1^1
	+n^0_{1,1}$, saying that
	$n^1_{2}$ is a homotopy for the relation giving  the bracket $n^0_{1,1}$ as the commutator between the operator
	$n_1^1$ and the product $m^0_2$.\anibal{probl\`eme de signe ! Doit \^etre dans le $d_\psi$.}


	\item[$\left(\mathsf{S}^2_{1}\right)$]
	$\partial\left(n^2_{1}\right)=\left[n^1_1, m^1_1\right]$
	saying that $n^2_{1}$ is a homotopy for the commutativity relation between $n^1_1$ and $m^1_1$.
\end{description}


\bigskip
\hrule
\bigskip









\subsection{Old}

\newpage
\noindent
\renewcommand{\arraystretch}{1.15}
\begin{tabularx}{\textwidth}{@{}l X@{}}
	\hline
	(\(\sM^0_{1}\)) &
	\( \displaystyle m^0_1 \circ_1 m^0_1 \) \\

	\hline
	(\(\sM^0_{2}\)) &
	\( \displaystyle
	m^0_1 \circ_1 m^0_2
	- m^0_2 \circ_1 m^0_1
	- m^0_2 \circ_2 m^0_1 \) \\

	\hline
	(\(\sM^1_1\)) &
	\( \displaystyle m^0_1 \circ m^1_1 + m^1_1 \circ m^0_1 \) \\

	(\(\sM^0_{1,1}\)) &
	\( \displaystyle
	m^0_1 \circ m^0_{1,1}
	- m^0_{1,1} \circ_1 m^0_1
	- m^0_{1,1} \circ_2 m^0_1 \) \\

	\hline
	(\(\sM^0_{3}\)) &
	\( \displaystyle m^0_2 \circ_1 m^0_2 - m^0_2 \circ_2 m^0_2 \) \\

	\hline
	(\(\sM^0_{1,2}\)) &
	\( \displaystyle
	m^0_{2} \circ_1 m^0_{1,1}
	+ (m^0_{2} \circ_2 m^0_{1,1}) \circ(12)
	- m^0_{1,1} \circ_2 m^0_2 \) \\

	(\(\sM^0_{1,1,1}\)) &
	\( \displaystyle
	(m^0_{1,1} \circ_1 m^0_{1,1}) \circ \big(\id + (123) + (132)\big) \) \\

	(\(\sM^1_{1,1}\)) &
	\( \displaystyle
	m^1_1 \circ m^0_{1,1}
	- m^0_{1,1} \circ_1 m^1_1
	- m^0_{1,1} \circ_2 m^1_1 \) \\

	(\(\sM^1_2\)) &
	\( \displaystyle
	- m^1_1 \circ_1 m^0_2
	+ m^0_2 \circ_1 m^1_1
	+ m^0_2 \circ_2 m^1_1
	+ m^0_{1,1}
	\) \\

	(\(\sM^2_1\)) &
	\( \displaystyle
	- m^1_1 \circ_1 m^1_1
	\) \\

	\hline
\end{tabularx}


\newpage



\noindent
\begin{tabularx}{\textwidth}{@{}l X@{}}
	\hline
	(\(\sM^0_{1}\)) &
	\( \displaystyle m^0_1 \circ_1 m^0_1 = 0 \) \\

	\hline
	(\(\sM^0_{2}\)) &
	\( \displaystyle
	m^0_1 \circ_1 m^0_2
	- m^0_2 \circ_1 m^0_1
	- m^0_2 \circ_2 m^0_1 = 0. \) \\

	(\(\sM^0_{1,1}\)) &
	\( \displaystyle [m^0_1, m^0_{1,1}] = 0. \) \\

	\hline
	(\(\sM^1_1\)) &
	\( \displaystyle [m^0_1, m^1_1] = 0. \) \\

	\hline
	(\(\sM^0_{3}\)) &
	\( \displaystyle m^0_2 \circ_1 m^0_2 - m^0_2 \circ_2 m^0_2 = 0. \) \\

	(\(\sM^0_{1,2}\)) &
	\( \displaystyle
	m^0_{2} \circ_1 m^0_{1,1}
	+ m^0_{2} \circ_2 m^0_{1,1} \circ(12)
	- m^0_{1,1} \circ_2 m^0_2 = 0 \) \\

	\hline
\end{tabularx}

\begin{align}
	\tag{(\(\sM^0_{3}\))}
	m^0_2 \circ_1 m^0_2 - m^0_2 \circ_2 m^0_2 &= 0
	\\
	\tag{(\(\sM^0_{1,2}\))}
	m^0_{2} \circ_1 m^0_{1,1}
	+ m^0_{2} \circ_2 m^0_{1,1} \circ(12)
	- m^0_{1,1} \circ_2 m^0_2 &= 0
	\\
	\tag{(\(\sM^0_{1,1,1}\))}
	\big(m^0_{1,1} \circ_1 m^0_{1,1}\big) \circ \big(\id + (123) + (132)\big) &= 0
	\\
	\tag{(\(\sM^1_{1,1}\))}
	[m^1_1, m^0_{1,1}] &= 0
	\\
	\tag{(\(\sM^1_2\))}
	[m^0_1, m^1_2] &=
	- m^1_1 \circ_1 m^0_2
	+ m^0_2 \circ_1 m^1_1
	+ m^0_2 \circ_2 m^1_1
	+ m^0_{1,1}
	+ n^1_2
	\\
	\tag{(\(\sM^2_1\))}
	[m^0_1, m^2_1] &=
	- m^1_1 \circ_1 m^1_1
	+ n^2_1
\end{align}


\begin{align}
	\tag{\(\sM^0_{3}\)}
	m^0_2 \circ_1 m^0_2 - m^0_2 \circ_2 m^0_2 &= 0
	\\
	\tag{\(\sM^0_{1,2}\)}
	m^0_{2} \circ_1 m^0_{1,1}
	+ m^0_{2} \circ_2 m^0_{1,1} \circ(12)
	- m^0_{1,1} \circ_2 m^0_2 &= 0
	\\
	\tag{\(\sM^0_{1,1,1}\)}
	(m^0_{1,1} \circ_1 m^0_{1,1} ) \circ \big(\id + (123) + (132)\big) &= 0
	\\
	\tag{\(\sM^1_{1,1}\)}
	[m^1_1, m^0_{1,1}] &= 0
\end{align}

\noindent\( (\sM^1_2) \):
\[
[m^0_1, m^1_2] =
- m^1_1 \circ_1 m^0_2
+ m^0_2 \circ_1 m^1_1
+ m^0_2 \circ_2 m^1_1
+ m^0_{1,1}
+ n^1_2.
\]

\( (\sM^2_1) \):
\[
[m^0_1, m^2_1] = - m^1_1 \circ_1 m^1_1 + n^2_1.
\]

\newpage
%\noindent\( (\sM^0_{1}) \):
%\( m^0_1 \) squares to zero:
%\[
%m^0_1 \circ_1 m_1^0 = 0,
%\]
%
%\noindent\( (\sM^0_2) \):
%\( m^0_1 \) is a derivation of \( m^0_2 \):
%\[
%[m^0_1, m^0_2] = 0.
%\]
%
%\noindent\( (\sM^1_1) \):
%\( n^1_1 \) is the obstruction to \( m^0_1 \) commuting with \( m^1_1 \):
%\[
%[m^0_1, m^1_1] = n^1_1.
%\]
%
%\noindent\( (\sM^0_{1,1}) \):
%\( n^0_{1,1} \) is the obstruction to \( m^0_1 \) being a derivation of \( m^0_{1,1} \):
%\[
%[m^0_1, m^0_{1,1}] = n^0_{1,1}.
%\]
%\noindent\( (\sM^2_1) \):
%\( n^2_1 \) is the obstruction to \( m^2_1 \) being a homotopy for the square-zero relation of \( m^1_1 \):
%\[
%[m^0_1, m^2_1] = - m^1_1 \circ_1 m^1_1 + n^2_1.
%\]
%\bigskip\hrule\bigskip
%
%\noindent\( (\sM^0_4) \):
%The operations \( m^0_2 \) and \( m^0_3 \) satisfy a coherence relation up to the homotopy \( m^0_4 \):
%\[
%[m^0_1, m^0_4] =
%m^0_2 \circ_1 m^0_3
%+ m^0_3 \circ_2 m^0_2
%+ m^0_2 \circ_2 m^0_3
%- m^0_3 \circ_3 m^0_2
%- m^0_3 \circ_1 m^0_2.
%\]
%\noindent\( (\sM^1_2) \):
%\( n^1_2 \) is the obstruction to \( m^1_2 \) being a homotopy realizing \( m^0_{1,1} \) as the commutator of \( m^1_1 \) with \( m^0_2 \):
%\[
%[m^0_1, m^1_2] =
%- [m^1_1, m^0_2] + m^0_{1,1} + n^1_2.
%%=\
%%- m^1_1 \circ_1 m^0_2
%%+ m^0_2 \circ_1 m^1_1
%%+ m^0_2 \circ_2 m^1_1
%%+ m^0_{1,1}
%%+ n^1_2.
%\]






%\begin{align}
%	\tag{\(\sM^0_{1}\)}
%	m^0_1 \circ m_1^0    &= 0, && m^0_1\text{ is a differential},
%	\\
%	\tag{\(\sM^0_2\)}
%	[m^0_1, m^0_2]         &= 0, && m^0_1\text{ is a derivation of the product } m^0_2,
%	\\
%	\tag{\(\sM^1_1\)}
%	[m^0_1, m^1_1]         &= 0, && m^0_1\text{ commutes with } m^1_1,
%	\\
%	\tag{\(\sM^0_{1,1}\)}
%	[m^0_1, m^0_{1,1}]     &= 0, && m^0_1\text{ is a derivation of } m^0_{1,1},
%\end{align}
%\hrule
%\begin{align}
%	\tag{\(\sM^0_{3}\)}
%	m^0_2 \circ_1 m^0_2 - m^0_2 \circ_2 m^0_2
%	&= 0, &&m^0_2 \text{ associativity},
%	\\
%	\tag{\(\sM^0_{1,2}\)}
%	m^0_{2} \circ_1 m^0_{1,1}
%	+ m^0_{2} \circ_2 m^0_{1,1} \circ(12)
%	- m^0_{1,1} \circ_2 m^0_2
%	&= 0, &&(m^0_2, m^0_{1,1})\text{ Leibniz},
%	\\
%	\tag{\(\sM^0_{1,1,1}\)}
%	(m^0_{1,1} \circ_1 m^0_{1,1} ) \circ \big(\id + (123) + (132)\big)
%	&= 0, &&m^0_{1,1}\text{ Jacobi},
%	\\
%	\tag{\(\sM^1_{1,1}\)}
%	[m^1_1, m^0_{1,1}]
%	&= 0, &&m^1_1\text{ is a derivation of } m^0_{1,1}.
%\end{align}

%\begin{align}
%	\tag{\(\sM^0_{3}\)}
%	m^0_2 \circ_1 m^0_2 - m^0_2 \circ_2 m^0_2 &= 0
%	\\
%	\tag{\(\sM^0_{1,2}\)}
%	m^0_{2} \circ_1 m^0_{1,1}
%	+ m^0_{2} \circ_2 m^0_{1,1} \circ(12)
%	- m^0_{1,1} \circ_2 m^0_2 &= 0
%	\\
%	\tag{\(\sM^0_{1,1,1}\)}
%	(m^0_{1,1} \circ_1 m^0_{1,1} ) \circ \big(\id + (123) + (132)\big) &= 0
%	\\
%	\tag{\(\sM^1_{1,1}\)}
%	[m^1_1, m^0_{1,1}] &= 0
%\end{align}

\medskip

\noindent
\((\sM^0_{3})\) expresses the associativity of \( m^0_2 \),
\((\sM^0_{1,2})\) encodes the Leibniz rule for \( m^0_2 \) and \( m^0_{1,1} \),
\((\sM^0_{1,1,1})\) is the Jacobi identity for \( m^0_{1,1} \), and
\((\sM^1_{1,1})\) states that \( m^1_1 \) is a derivation of \( m^0_{1,1} \).

\noindent
\begin{tabularx}{\textwidth}{>{\raggedright}m{.5cm} >{\raggedright\arraybackslash}m{7.2cm} X}
	\textbf{(\(\sM^0_{3}\))} &
	$\displaystyle m^0_2 \circ_1 m^0_2 - m^0_2 \circ_2 m^0_2 = 0$ &
	This relation expresses the associativity of $m^0_2$.
	\\[1.2em]
	\textbf{(\(\sM^0_{1,2}\))} &
	$\displaystyle m^0_{2} \circ_1 m^0_{1,1}
	+ m^0_{2} \circ_2 m^0_{1,1} \circ (12)
	- m^0_{1,1} \circ_2 m^0_2 = 0$ &
	This relation encodes the Leibniz rule for $m^0_2$ and $m^0_{1,1}$. If needed, the explanation can continue on multiple lines.
	\\[1.2em]
	\textbf{(\(\sM^0_{1,1,1}\))} &
	$\displaystyle (m^0_{1,1} \circ_1 m^0_{1,1}) \circ \big(\id + (123) + (132)\big) = 0$ &
	This relation is the Jacobi identity for $m^0_{1,1}$.
	\\[1.2em]
	\textbf{(\(\sM^1_{1,1}\))} &
	$\displaystyle [m^1_1, m^0_{1,1}] = 0$ &
	This relation states that $m^1_1$ is a derivation of $m^0_{1,1}$.
\end{tabularx}



\subsubsection{\( \bvbox \)}

\subsection{\( \bvbox_\infty(2) \)-algebras}

\noindent\( (\sM^0_{1}) \):
\( m^0_1 \) squares to zero:
\[
m^0_1 \circ_1 m_1^0 = 0.
\]

\noindent\( (\sM^0_2) \):
\( m^0_1 \) is a derivation of \( m^0_2 \):
\[
[m^0_1, m^0_2] = 0.
\]

\medskip\noindent\( (\sM^1_1) \):
\( n^1_1 \) is the obstruction to \( m^0_1 \) commuting with \( m^1_1 \):
\[
[m^0_1, m^1_1] = n^1_1.
\]

\noindent\( (\sM^0_{1,1}) \):
\( n^0_{1,1} \) is the obstruction to \( m^0_1 \) being a derivation of \( m^0_{1,1} \):
\[
[m^0_1, m^0_{1,1}] = n^0_{1,1}.
\]

\bigskip\hrule\bigskip

\noindent\( (\sM^0_{3}) \):
\( m^0_3 \) is the associator for \( m^0_2 \):
\[
[m^0_1, m^0_3] = m^0_2 \circ_1 m^0_2 - m^0_2 \circ_2 m^0_2.
\]

\noindent\( (\sM^0_{1,2}) \):
\( n^0_{1,2} \) is the obstruction to \( m^0_{1,2} \) being a leibniziator for \( m^0_2 \) and \( m^0_{1,1} \):
\[
[m^0_1, m^0_{1,2}] =
m^0_{2} \circ_1 m^0_{1,1}
+ (m^0_{2} \circ_2 m^0_{1,1})^{(12)}
- m^0_{1,1} \circ_2 m^0_2
+ n^0_{1,2}.
\]\anibal{Doesn't seem to match the tree representation of Leibniz on page 6 of JazzClub.pdf}

\medskip\noindent\( (\sM^0_{1,1,1}) \):
\( n^0_{1,1,1} \) is the obstruction to \( m^0_{1,1,1} \) being a jacobiator for \( m^0_{1,1} \):
\[
[m^0_1, m^0_{1,1,1}] =
-m^0_{1,1} \circ_1 m^0_{1,1}
- (m^0_{1,1} \circ_1 m^0_{1,1})^{(123)}
- (m^0_{1,1} \circ_1 m^0_{1,1})^{(321)}
+ n^0_{1,1,1}.
\]

\medskip\noindent\( (\sM^1_{1,1}) \):
\( n^1_{1,1} \) is the obstruction to \( m^1_{1,1} \) being a derivator for \( m^1_1 \) and \( m^0_{1,1} \):
\[
[m^0_1, m^1_{1,1}] =
-[m^1_1, m^0_{1,1}]
+ n^1_{1,1}.
\]

%\noindent\( (\sN^1_{1,1}) \):
%\( n^1_{1,1} \) is a homotopy between the commutator of \( n^1_1 \) with \( m^0_{1,1} \) and the commutator of \( m^1_1 \) with \( n^0_{1,1} \):
%\[
%[m^0_1, n^1_{1,1}] =
%[n^1_1, m^0_{1,1}] - [m^1_1, n^0_{1,1}].
%= n^1_1 \circ_1 m^0_{1,1}
%- m^0_{1,1} \circ_1 n^1_1
%- m^0_{1,1} \circ_2 n^1_1
%- m^1_1 \circ_1 n^0_{1,1}
%+ n^0_{1,1} \circ_1 m^1_1
%+ n^0_{1,1} \circ_2 m^1_1.
%\]

\noindent\( (\sM^1_2) \):
\( n^1_2 \) is the obstruction to \( m^1_2 \) being a homotopy realizing \( m^0_{1,1} \) as the commutator of \( m^1_1 \) with \( m^0_2 \):
\[
[m^0_1, m^1_2] =
- [m^1_1, m^0_2] + m^0_{1,1} + n^1_2.
%=\
%- m^1_1 \circ_1 m^0_2
%+ m^0_2 \circ_1 m^1_1
%+ m^0_2 \circ_2 m^1_1
%+ m^0_{1,1}
%+ n^1_2.
\]

%\noindent\( (\sN^1_2) \):
%\( n^1_2 \) is a homotopy for the relation expressing \( n^0_{1,1} \) as the commutator of \( n^1_1 \) with \( m^0_2 \):
%\[
%[m^0_1, n^1_2] = [n^1_1, m^0_2] - n^0_{1,1}.
%=\
%n^1_1 \circ_1 m^0_2
%- m^0_2 \circ_1 n^1_1
%- m^0_2 \circ_2 n^1_1
%+ n^0_{1,1}.
%\]

\noindent\( (\sM^2_1) \):
\( n^2_1 \) is the obstruction to \( m^2_1 \) being a homotopy for the square-zero relation of \( m^1_1 \):
\[
[m^0_1, m^2_1] = - m^1_1 \circ_1 m^1_1 + n^2_1.
\]

%\noindent\( (\sN^2_1) \): Missing?

\bigskip\hrule\bigskip

\noindent\( (\sM^0_4) \):
The operations \( m^0_2 \) and \( m^0_3 \) satisfy a coherence relation up to the homotopy \( m^0_4 \):
\[
[m^0_1, m^0_4] =
m^0_2 \circ_1 m^0_3
+ m^0_3 \circ_2 m^0_2
+ m^0_2 \circ_2 m^0_3
- m^0_3 \circ_3 m^0_2
- m^0_3 \circ_1 m^0_2.
\]

\noindent\( (\sM^0_{1,1,1,1}) \):
\( n^0_{1,1,1,1} \) measures the obstruction to \( m^0_{1,1,1,1} \) implementing the higher Jacobi identity for the shifted \( L_\infty \)-structure defined by \( m^0_{1,1} \) and \( m^0_{1,1,1} \):
\begin{align*}
	[m^0_1, m^0_{1,1,1,1}] = &-
	(m^0_{1,1} \circ_1 m^0_{1,1,1})^{\id + (1234) + (13)(24) + (4321)} \\ &-
	(m^0_{1,1,1} \circ_1 m^0_{1,1})^{\id + (23) + (234) + (123) + (1342) + (13)(24)}
	+ n^0_{1,1,1,1}.
\end{align*}

\bigskip\hrule\bigskip

\newpage
\begin{definition}
	A \emph{\( \bvbox_\infty \)-algebra} is a graded vector space \( A \) equipped with a familiy of operations
	\begin{align*}
		m_{p_1,\dots,p_k}^t &\colon
		A^{\ot p_1} \ot \dotsb \ot A^{\ot p_k}
		\longrightarrow A~, &&
		t \geqslant 0~, \  k \geqslant 1~, \  p_1,\dots, p_k \geqslant 1~
	\end{align*}
	of degrees
	\[
	| m_{p_1,\dots,p_k}^t |  =  p_1 + \cdots + p_k + k + 2t - 3.
	\]
	Each of these satisfies the following block and shuffle symmetry properties:

	\medskip\noindent(1)
	For any permutation \( \sigma \in \sym_k \):
	\[
	m_{p_{\sigma(1)},\dots,p_{\sigma(k)}}^t = \bar\sigma \cdot m_{p_1,\dots,p_k}^t~,
	\]
	where \( \bar\sigma \) is the image of \( \sigma \) in the block inclusion \( \sym_k \to \sym_{p_1+\dots+p_k} \).

	\medskip\noindent(2)
	For any \( i \in \set{1,\dots,k}\) and \( j \in \set{1,\dots,p_i-1} \):
	\begin{align*}
		&\sum_{\mathclap{\sigma \in \Sh(j,\, p_i - j)}} \ \sign(\sigma) \, \underline\sigma \cdot
		m_{p_1,\dots,p_k}^t  =  0~,
	\end{align*}
	where \( \underline\sigma \) is the image of \( \sigma \) in the natural inclusion \( \sym_{p_i} \to \sym_{p_1+\dots+p_k} \).

	The only relations are those making \( A, \set{m^0_\ell}_{\ell \geq 0} \) into a \( C_\infty \) algebra.


	\begin{align*}\tag{$\sR^t_{p_1,\dots,p_k}$}
		&\sum_{\substack{
				0 \leqslant s \leqslant t \\
				I \sqcup J  =  \{1, \dots, k\} \\
				I  =  \{i_1, \dots, i_a\} \\
				J  =  \{j_1, \dots, j_b\}
		}}
		\sum_{\substack{
				q_1, \dots, q_a \geqslant 1 \\
				(q_1, \dots, q_a) \leqslant (p_1, \dots, p_{i_a})
		}}
		(-1)^{\varepsilon + \varepsilon' + \varepsilon''}
		m^s_{p', p_{j_1}, \dots, p_{j_b}} (
		m^{t-s}_{\frac{p_1, \dots, p_{i_a}}{q_1, \dots, q_a}}
		(w_{i_1} \ot \dots \ot w_{i_a}) \ot w_{j_1} \ot \dots \ot w_{j_b}
		) \\
		&\quad
		- \sum_{i  =  1}^k
		\sum_{1 \leqslant p \leqslant p_i - 1}
		(-1)^{\varepsilon'''}
		m^{t-1}_{p_1, \dots, p, p_i - p, \dots, p_k} (
		w_1 \ot \dots \ot a_{[P_i + 1, P_i + p]} \ot a_{[P_i + p + 1, P_{i+1}]} \ot \dots \ot w_k
		) \\
		&\quad
		- (t + k - 1)\, n^t_{p_1, \dots, p_k}  =  0~,
	\end{align*}
\end{definition}


\subsection{Product}

This might be reducible since it is symmetric?

\begin{align*}
	&\theta_+ \ot \theta_+
	\mapsto \theta_+ (\id \ot \id), \\
	%%%
	&\theta_+ \ot \theta_\mu
	\mapsto \theta_\mu (\id \ot \id) + c\theta_+( \partial_\mu \ot \id + \id \ot \partial_\mu), \\
	%%%
	&\theta_+ \ot \theta_-
	\mapsto -c\theta_\mu (\id \ot \partial^\mu), \\
	%%%
	&\theta_+ \ot c\theta_\mu
	\mapsto c\theta_\mu (\id \ot \id), \\
	%%%
	&\theta_+ \ot c\theta_-
	\mapsto c\theta_- (\id \ot \id), \\
	%%%
	&\theta_\mu \ot \theta_+
	\mapsto \theta_\mu (\id \ot \id) + c\theta_+( \partial_\mu \ot \id + \id \ot \partial_\mu), \\
	%%%
	&\theta_\mu \ot \theta_\nu
	\mapsto
	c\theta_\nu ( \partial_\mu \ot \id + 2\, \id \ot \partial_\mu ) \\
	&\phantom{(\theta_\mu, \theta_\nu) \mapsto}
	- c\theta_\mu ( \id \ot \partial_\nu + 2\, \partial_\nu \ot \id ) \\
	&\phantom{(\theta_\mu, \theta_\nu) \mapsto}
	+ c\theta_\rho \eta_{\mu\nu} ( \partial^\rho \ot \id - \id \ot \partial^\rho ), \\
	%%%
	&\theta_\mu \ot \theta_-
	\mapsto c\theta_- (\id \ot \partial_\mu), \\
	%%%
	&\theta_\mu \ot c\theta_\nu
	\mapsto -c\theta_- \eta_{\mu \nu} (\id \ot \id), \\
	%%%
	&\theta_- \ot \theta_+
	\mapsto -c\theta_\mu (\partial^\mu \ot \id), \\
	%%%
	&\theta_- \ot \theta_\mu
	\mapsto c\theta_- (\partial_\mu \ot \id),\\
	%%%
	&c\theta_\mu \ot \theta_+
	\mapsto c\theta_\mu (\id \ot \id),\\
	%%%
	&c\theta_\mu \ot \theta_\nu
	\mapsto -c\theta_- \eta_{\mu \nu} (\id \ot \id),\\
	%%%
	&c\theta_- \ot \theta_+
	\mapsto c\theta_-(\id \ot \id).
\end{align*}


%\begin{align*}
%	&\theta_+ \ot \theta_+
%	\mapsto \theta_+ (\id \ot \id), \\
%	%%%
%	&\theta_+ \ot \theta_\mu
%	\mapsto \theta_\mu (\id \ot \id) + c\theta_+( \partial_\mu \ot \id + \id \ot \partial_\mu), \\
%	%%%
%	&\theta_+ \ot \theta_-
%	\mapsto -c\theta_\mu (\id \ot \partial^\mu), \\
%	%%%
%	&\theta_+ \ot c\theta_\mu
%	\mapsto c\theta_\mu (\id \ot \id), \\
%	%%%
%	&\theta_+ \ot c\theta_-
%	\mapsto c\theta_- (\id \ot \id), \\
%	%%%
%	&\theta_\mu \ot \theta_\nu
%	\mapsto
%	c\theta_\nu ( \partial_\mu \ot \id + 2\, \id \ot \partial_\mu ) \\
%	&\phantom{(\theta_\mu, \theta_\nu) \mapsto}
%	- c\theta_\mu ( \id \ot \partial_\nu + 2\, \partial_\nu \ot \id ) \\
%	&\phantom{(\theta_\mu, \theta_\nu) \mapsto}
%	+ c\theta_\rho \eta_{\mu\nu} ( \partial^\rho \ot \id - \id \ot \partial^\rho ), \\
%	%%%
%	&\theta_\mu \ot \theta_-
%	\mapsto c\theta_- (\id \ot \partial_\mu), \\
%	%%%
%	&\theta_\mu \ot c\theta_\nu
%	\mapsto -c\theta_- \eta_{\mu \nu} (\id \ot \id),
%\end{align*}



\newpage
%\begin{align}
%	&d(\theta_+ \ot \lambda)  =
%	\theta_1 \ot \partial_0 \lambda \ -\
%	\sum_{i = 1}^{d-1} \theta_i \ot \partial_i \lambda \ +\
%	c\theta_+ \ot \square \lambda &
%	\begin{pmatrix}
	%		\partial^\mu \lambda \\ \square\lambda
	%	\end{pmatrix} \\
%	& d(\theta_\mu \ot A^\mu)  =
%	c\theta_- \ot \partial_\mu A^\mu \ +\
%	c\theta_\mu \ot \square  A^\mu &
%	\begin{pmatrix}
	%		\partial \cdot A \\ \square A
	%	\end{pmatrix} \\
%	& d(c\theta_+ \ot \varphi)  =
%	c\theta_- \ot (-\varphi) \ +\
%	c\theta_1 \ot \partial_0 \varphi \ -\
%	\sum_{i = 1}^{d-1} \theta_i \ot \partial_i \varphi &
%	\begin{pmatrix}
	%		-\varphi \\ \partial^\mu \varphi
	%	\end{pmatrix} \\
%	& d(\theta_- \ot \phi)  =  c\theta_- \ot \square \phi &
%	\begin{pmatrix}
	%		\square \phi
	%	\end{pmatrix} \\
%	& d(c\theta_\mu \ot E^\mu)  =
%	c\theta_- \ot (-\partial_\mu E^\mu) &
%	\begin{pmatrix}
	%		- \partial \cdot E
	%	\end{pmatrix}
%\end{align}


jacobi


%\begin{align}
%	(\theta_+, \theta_+, \theta_-)
%	&\mapsto \theta_+,  \\
%	(\theta_+, \theta_+, c\theta_-)
%	&\mapsto -c\theta_+ (\id \ot \id \ot \id),  \\
%	(\theta_+, \theta_\mu, c\theta_\nu)
%	&\mapsto c\theta_+ \eta_{\mu\nu} (\id \ot \id \ot \id),  \\
%	(\theta_+, \theta_\mu, \theta_-)
%	&\mapsto \theta_\mu (\id \ot \id \ot \id)
%	+ c\theta_+ (\partial_\mu \ot \id \ot \id),  \\
%	(\theta_+, c\theta_+, \theta_-)
%	&\mapsto c\theta_+ (\id \ot \id \ot \id),  \\
%	(\theta_+, \theta_-, \theta_-)
%	&\mapsto \theta_- (\id \ot \id \ot \id),  \\
%	(\theta_+, \theta_-, c\theta_\mu)
%	&\mapsto c\theta_\mu (\id \ot \id \ot \id),  \\
%	(\theta_+, \theta_-, c\theta_-)
%	&\mapsto c\theta_- (\id \ot \id \ot \id),  \\
%	(\theta_\mu, \theta_-, \theta_-)
%	&\mapsto 2 c\theta_- ( \partial_\mu \ot \id \ot \id
%	+ \id \ot \id \ot \partial_\mu ),  \\
%	(\theta_\mu, c\theta_\nu, \theta_-)
%	&\mapsto -c\theta_- \eta_{\mu\nu} (\id \ot \id \ot \id),  \\
%	(c\theta_+, \theta_-, \theta_-)
%	&\mapsto c\theta_- (\id \ot \id \ot \id). \tag{A10}
%\end{align}
